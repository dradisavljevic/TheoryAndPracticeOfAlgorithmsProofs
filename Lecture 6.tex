\documentclass[a4paper]{article}

\usepackage{amsmath,amsthm,amssymb, titling, mdframed}

\newcommand{\Lim}[1]{\raisebox{0.5ex}{{$\displaystyle \lim_{#1}\;$}}}
\title{\vspace{-2cm}Lecture VI\vspace{-2cm}}
\date{}

\newmdtheoremenv{theorem}{Theorem}
\newmdtheoremenv{definition}{Definition}
\newmdtheoremenv{corollary}{Corollary}
\newmdtheoremenv{algorithm}{Algorithm}
\newmdtheoremenv{lemma}{Lemma}

\begin{document}
\maketitle
\section{Lecture Summary}
\textbf{Theorem 6 from Lecture 5} is the starting point for the AKS Algorithm: $n$ is prime if and only if $(X + a)^{n} \equiv (X^{n} + a)$ mod $n$. From there they figured that it suffices to check the following: \\
$$(X + a)^{n}\ mod\ (n, X^{r} - 1)\ and\ (X^{n} + a)= mod (n, X^{r} - 1)$$
If $n$ is prime, then this is clearly satisfied. \\
When varying $a$, one has to perform the check for several values. If all these tests are fulfilled then $n$ is either a prime or a prime power.
\begin{algorithm}
\textbf{AKS -} \\
Input: Odd integer $n > 1$ \\
if ($n$ is of the form $a^{b}$, $b > 1$) output COMPOSITE; \\
$r := 2$; \\
while $( r < n )$ \{\\
\indent if ( $gcd(n,r) \neq 1$ ) output COMPOSITE; \\
\indent if ( $r$ is prime ) then \\
\indent \indent let $q$ be the largest prime factor of $r - 1$; \\
\indent \indent if $( q \geq 4 \sqrt{r} \log(n) )$ \\
\indent \indent and $( n^{\frac{r - 1}{q}} \not \equiv 1$ mod $r )$ then \\
\indent \indent \indent break; \\
$r := r + 1$; \\
\} \\
for $a = 1$ to $\lfloor 2 \sqrt{r} \log{n} \rfloor$ do \\
\indent if $( (X + a)^{n} \not \equiv (X^{n} + a)$ mod $(n, X^{r} - 1) )$ then\\
\indent output COMPOSITE; \\
output PRIME;
\end{algorithm}


\begin{lemma}
Let $p$ be a prime, let $a \in \mathbb{Z}_{p}^{*}$, and let $d = ord(a)$. Then we have\\
(1) $d$ divides $p - 1$. \\
(2) If $q$ is a prime such that $q|(p - 1)$ but $q^{2}$ does not divide $p - 1$, then for $s = \frac{p - 1}{q}$ we have
\begin{center}
$q|d$ if and only if $a^{s} \not \equiv 1$ mod $p$.
\end{center}
\end{lemma}


\begin{theorem}
\textbf{Prime Number Theorem -} The following assertion holds: $\Lim{x \to \infty}{\frac{\pi (x)}{x/ ln\ x}} = 1$
\end{theorem}


\begin{theorem}
For all $x > 2$ we have 
\begin{center}
$\frac{x}{6\log{x}} \leq \pi (x) \leq \frac{8x}{\log(x}$
\end{center}
\end{theorem}


\begin{theorem}
\textbf{Fouvry's Theorem -} There is a constant $c > 0$ and a real number $x_{0} > 0 $ such that for all $x \geq x_{0}$ we have $\pi^{**} (x) \geq c \cdot \frac{x}{\log{x}}$.
\end{theorem}


\begin{lemma}
There exists an $n_{0}$ such that for all $n \geq n_{0}$ there is a prime $r$ satisfying\\
(1) $r \leq 4096 \cdot (\log(n))^{6}$ \\
(2) either $r$ divides $n$ or $r$ is $n$-good.
\end{lemma}


\begin{theorem}
The running time of Algorithm AKS is $O\left( (\log(n))^{17} \right)$
\end{theorem}
\end{document}