\documentclass[a4paper]{article}

\usepackage{amsmath,amsthm,amssymb, titling, mdframed}

\title{\vspace{-2cm}Lecture XIII\vspace{-2cm}}
\date{}

\newmdtheoremenv{theorem}{Theorem}
\newmdtheoremenv{definition}{Definition}
\newmdtheoremenv{corollary}{Corollary}
\newmdtheoremenv{algorithm}{Algorithm}
\newmdtheoremenv{lemma}{Lemma}

\begin{document}
\maketitle
\section{Lecture Summary}
\begin{definition}
The value of an AS-number $x = (x_{n-1}, \ldots, x_{0}) \in \{ 0, 1 \}^{n}$ is defined as
$$V_{AS}(x) =_{df} (-1)^{x_{n-1}}(x_{n-2}2^{n-2} + \cdots + x_{1}2 + x_{0}).$$
\end{definition}


\begin{theorem}
There is a log-space uniform family $(C_{n})_{n \in \mathbb{N}}$ of Boolean circuits achieving simultaneously depth $O(\log{n})$ and size $O(n)$ such that $C_{n}$, on input two $n$-bit AS-numbers $x$, $y$, correctly computes $x + y$.
\end{theorem}


\begin{theorem}
There is no family $(C_{n})_{n \in \mathbb{N}}$ of Boolean circuits over the basis $\{ \lnot, \land, \lor \}$ for binary addition such that $C_{n}$ computes the sum of any two $n$-bit numbers, which achieves depth $o(\log{n})$.
\end{theorem}


\begin{definition}
\textbf{Fermat Ring -} Let $m \in \mathbb{N} \backslash \{ 0 \}$. \\
(i) $\mathbb{Z}_{m}$ is called Fermat ring provided $m = 4^{k} + 1$ for some $k \in \mathbb{N}, k \geq 1$. \\
(ii) Let $\mathbb{Z}_{m}$ be a Fermat ring and let $x \in \mathbb{Z}_{m}$. A radix-$4$ representation for $x$ is any vector $x = (x_{n}, \ldots, x_{0})$ with $-3 \leq x_{i} \leq 3$ such that 
$$|x|_{4} = x_{n}4^{n} + x_{n - 1}4^{n - 1} + \cdots + x_{0} \equiv x\ mod\ m.$$
\end{definition}


\begin{theorem}
There exists a log-space uniform family $(C_{n})_{n \in \mathbb{N}}$ of Boolean circuits achieving simultaneously depth $O(1)$ and size $O(n)$ such that $C_{n}$, on input any two $3(n + 1)$ bit radix-$4$ numbers $x$, $y$ correctly computes $x + y$ mod $m$, where $m = 4^{n} + 1$.
\end{theorem}


\begin{corollary}
There exists a log-space uniform family $(C_{n})_{n \in \mathbb{N}}$ of Boolean circuits achieving simultaneously depth $O(1)$ and size $O(n)$ such that $C_{n}$, on input two $3(n + 1)$ bit radix-$4$ numbers $x$, $y$ correctly computes $x - y$ mod $m$, where $m = 4^{n} + 1$.
\end{corollary}


\begin{samepage}
\begin{theorem}
(1) Let $a_{x} = (x_{2n}, \ldots, x_{0})$ be the binary representation of a number $a \in \{ 0, \ldots, 2^{2n+1} - 1 \}$, and let $m = 4^{n + 1} + 1$. Then $a_{x}$ can be transformed into radix-$4$ representation of $a$ by a log-space uniform family $(C_{n})_{n \in \mathbb{N}}$ of Boolean circuits that has simultaneously constant depth and linear size. \\
(2) There is a log-space uniform family $(C_{n})_{n \in \mathbb{N}}$ of Boolean circuits that has simultaneously depth $O(\log{n})$ and size $O(n)$ such that $C_{n}$ computes on input a radix-$4$ representation $(x_{n}, \ldots, x_{0}$ of any number $a \in \{ 0, \ldots, m - 1 \}$ the binary representation for $a$.
\end{theorem}
\end{samepage}


\begin{theorem}
There exists a log-space uniform family $(C_{n})_{n \in \mathbb{N}}$ of Boolean circuits achieving simultaneously depth $O(\log{n})$ and size $O(n^{2})$ such that $C_{n}$, on input $n$ numbers $a_{1}, \ldots, a_{n}$ having at most $n$ bits each, correctly computes $a_{1} + \cdots + a_{n}$.
\end{theorem}



\section{Exercises}
\noindent \textbf{Exercise 13.1}  Repeat the computation presented in \textbf{Example 13.6} for the case that $a = 13$ is given as $(0, 3, 1)$ as input.
\begin{proof}
Let $m = 4^{2} + 1 = 17$ and let $a=13$ be given as input as $(0, 3, 1)$.
Thus, we obtain:
\begin{center}
$\vert x^{+} \vert = (0,0,1,1,0,1)$\\
$\vert x^{-} \vert = (0,0,0,0,0,0)$
\end{center}
and $\vert x^{+} \vert - \vert x^{-} \vert =(0,0,1,1,0,1) =13$ (written as AS-number) (Since we first negate the first number and add it to second, then negating the result, this gives us the first number since the second one is all 0s). So again, for this example, we can omit the computation of other candidates, since 13 is between 0 and our $m$.
\end{proof}



\noindent\rule{12cm}{0.4pt}\\
\noindent \textbf{Exercise 13.2} Prove that there is a uniform family $\left(C_{n}\right)_{n \in \mathbb{N}}$ of Boolean circuits simultaneously achieving depth $O(1)$ and size $O(n)$ which on input any number b given in a radix-4 representation $(x_{n},  \ldots , x_{0})$ computes the radix-4 representation of the product $b2^{l}$, where $l \leq 2n$ is any fixed number.
\begin{proof}
Let's consider circuit stated in \textbf{Theorem 5}, the first part converts $a_{i}$ where $i = 1, \ldots, n$ into a radix-4 representation. SInce this requires n sub-circuits having depth $O(1)$ and size $O(n)$. In \textbf{Theorem 4} we have that conversion from binary into radix-4 can be done in constant depth with linear size. Since the operation we are interested in is multiplication with a power of two, where power of two is fixed such that it is $l \leq 2n$, we can say that $b2^{l} < 4^{n+1}$, so proof of this exercise comes directly from \textbf{Theorem 4} assertion (1).
\begin{center}
$x_{i} = 2y_{2i+1}+y_{2i}$ where $y_{i} \in {-1, 0, +1}$
\end{center}
\end{proof}
\end{document}