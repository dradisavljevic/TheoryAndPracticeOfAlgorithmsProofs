\documentclass[a4paper]{article}

\usepackage{amsmath,amsthm,amssymb,graphicx,titling, mdframed}

\title{\vspace{-2cm}Lecture IV\vspace{-2cm}}
\date{}

\newmdtheoremenv{theorem}{Theorem}
\newmdtheoremenv{definition}{Definition}
\newmdtheoremenv{corollary}{Corollary}
\newmdtheoremenv{algorithm}{Algorithm}
\newmdtheoremenv{lemma}{Lemma}

\begin{document}
\maketitle
\section{Lecture Summary}
\begin{definition}
Let $\mathbb{F} = (F, +, \cdot)$ be any Abelian field. We call the minimum number $c \in \mathbb{N}^{+}$ for which $\sum_{i = 1}^{c} 1 = 0$ the characteristic of $\mathbb{F}$ provided it exists. If there is no $c \in \mathbb{N}^{+}$ such that $\sum_{i=1}^{c} 1 = 0$ then we define the characteristic of $\mathbb{F}$ to be 0.
\end{definition}


\begin{theorem}
Let $\mathbb{F}_{q}$ be any finite Abelian field. Then char($\mathbb{F}_{q}$) must be a prime number. The characteristic of an infinite Abelian field is either a prime number or it is zero. Moreover, if the characteristic of an Abelian field $\mathbb{F}$ is zero then $F$ must be infinite.
\end{theorem}


\begin{corollary}
Let $\mathbb{F}_{q}$ be any Abelian field. Then the field integers are closed under multiplication.
\end{corollary}


\begin{theorem}
In every Abelian field $\mathbb{F}$ of characteristic $p$, where $p \in \mathbb{N}^{+}$, the field integers form a subfield of order $p$ isomorphic to the field $\mathbb{Z}_{p}$.
\end{theorem}


\begin{theorem}
In every Abelian field $\mathbb{F}$ of characteristic $p$, $p \in \mathbb{N}^{+}$, the equation
$$(x - a)^{p} = x^{p} - a^{p}$$
is satisfied for all $x, a \in F$.
\end{theorem}


\begin{corollary}
In every finite Abelian field $\mathbb{F}$ of characteristic $p$ there does not exist any element that has an order $pk$, where $k \in \mathbb{N}^{+}, k > 1$.
\end{corollary}


\begin{corollary}
Let $w_{1}, \ldots, w_{k}$ be any elements of an Abelian field $\mathbb{F}$ of characteristic $p$. Then we have $\left( \sum_{i=1}^{k} w_{i} \right)^{p^{n}} = \left( \sum_{i=1}^{k} w_{i}^{p^{n}} \right)$ for all $n \in \mathbb{N}$.
\end{corollary}


\begin{corollary}
Let $\mathbb{F}$ be any Abelian field of characteristic $p$, and let $k$ be any field integer. Then we have $k^{p^{n}} = k$ for all $n \in \mathbb{N}$.
\end{corollary}


\begin{theorem}
Let $\mathbb{F}$ be any Abelian field of characteristic p. Then for every $a \in F$ we have the following: The element $a$ is a field integer in $\mathbb{F}$ iff it is a solution of the equation $x^{p} - x = 0$.
\end{theorem}


\begin{corollary}
Let $\mathbb{F}$ be any Abelian field of characteristic $p$, and let $w \in F$ be such that $w$ is not a field integer. Then we have $w^{p} \neq w$.
\end{corollary}


\begin{theorem}
Let $\mathbb{F}$ be any Abelian field of characteristic $p$, let $f \in \mathbb{Z}_{p}[x]$ and let $w \in F$ be such that $f(w) = 0$. Then we have $f(w^{p^{n}}) = 0$ for all $n \in \mathbb{N}$.
\end{theorem}


\begin{theorem}
Let $\mathbb{F}$ be any finite Abelian field of characteristic $p$, let $w \in F$ be any element of order $n$. Furthermore, let $m$ be the order of $p$ in $\mathbb{Z}_{n}^{*}$. Then we have $w^{p^{m}} = w$, and the $m$ elements $w, w^{p}, w^{p^{2}}, \ldots, w^{p^{m-1}}$ are pairwise distinct.
\end{theorem}


\begin{theorem}
Let $w$ be an element of order $n$ in a finite Abelian field $\mathbb{F}_{q}$ of characteristic $p$, and let $m$ be the order of $p$ in $\mathbb{Z}_{n}^{*}$. Then the coefficients of the $m$th degree polynomial $f(x) = \prod_{i=0}^{m-1}\left( x - w^{p^{i}} \right)$ are field integers. Furthermore, $f$ is irreducible in $\mathbb{Z}_{p}[x]$.
\end{theorem}


\begin{theorem}
If $w$ is a field element of degree $m$ in a finite Abelian field $\mathbb{F}_{q}$ of characteristic $p$ then the polynomials over the field integers of $\mathbb{F}_{q}$ of degree less than $m$ in $w$ form a subfield $\mathbb{F}_{q}$ that has order $p^{m}$.
\end{theorem}


\begin{theorem}
The order of every finite Abelian field is a power of its characteristic.
\end{theorem}



\section{Exercises}
\noindent \textbf{Exercise 4.1} Find an irreducible polynomial of degree 2 over $\mathbb{Z}_{7}$.
\begin{proof}
Polynomials of degree 2 or 3 are irreducible iff they have no roots\footnote{Theorem and proof that polynomials of degree 2 or 3 are irreducible iff they have no roots - http://mathonline.wikidot.com/reducible-and-irreducible-polynomials-over-a-field}. If we take a look at elements in $\mathbb{Z}_{7}$, and square them, we can see that there are no nonzero squares other than 1, 2 and 4 mod 7. So, the following polynomials have no solutions in $\mathbb{Z}_{7}$:
\begin{center}
$x^2 - 3 = 0$\\
$x^2 - 5 = 0$\\
$x^2 - 6 = 0$	
\end{center}
\end{proof}



\noindent\rule{12cm}{0.4pt}\\
\noindent \textbf{Exercise 4.2} Find out whether or not there is a finite field having 8 elements. In case your answer is affirmative, construct such a finite field.
\begin{proof}
Since 8 is a power of a prime number ($2^3$), we know that finite field with 8 elements exists. Since 8 is a power of 2, we need a 3rd degree polynomial $f$ which is irreducible over $\mathbb{Z}_{2}[x]$. Since polynomials of degree 2 or 3 are irreducible iff they have no roots, so we need to find a polynomial that has no roots. We take $x^3 + x + 1$, if we test it for elements of $\mathbb{Z}_{2}[x]$
\begin{center}
$p(0) = 1 \neq 0$\\
$p(1) = 1 \neq 0$	
\end{center}
Meaning the said polynomial is irreducible and can be taken for the purpose (other option would be $x^3 + x^2 + 1$). We form the field by using $\frac{\mathbb{Z}_{2}[x]}{x^3 + x + 1}$. In this field, if we introduce a relation that $x^3 = x + 1$ we can construct a table of elements (elements are possible residuals of our polynomial $x^3 + x + 1$) below:
\begin{center}
\centering
\resizebox{\textwidth}{!}{\begin{tabular}{ c c c c c c c c c  }
\textbf{Product mod $p(x)$} & \textbf{0} & \textbf{1} & \textbf{$x$} & \textbf{$x$+1} & \textbf{$x^2$} & \textbf{$x^2$ + 1} & \textbf{$x^2$ + $x$} & \textbf{$x^2+x+1$} \\ 
\textbf{0} & 0 & 0 & 0 & 0 & 0 & 0 & 0 & 0 \\
\textbf{1} & 0 & 1 & $x$ & $x+1$ & $x^2$ & $x^2$ + 1 & $x^2$ + $x$ & $x^2$ + $x$ + 1 \\
\textbf{$x$} & 0 & $x$ & $x^2$ & $x^2+x$ & $x$ + 1 & 1 & $x^2$ + $x$ + 1 & $x^2$ + 1 \\
\textbf{$x$ + 1} & 0 & $x$ + 1 & $x^2$ + $x$ & $x^2$ + 1 & $x^2$ + $x$ + 1 & $x^2$ & 1 & $x$ \\
\textbf{$x^2$} & 0 & $x^2$ & $x$ + 1 & $x^2$ + $x$ + 1 & $x^2$ + $x$ & $x$ & $x^2$ + 1 & 1 \\
\textbf{$x^2$ + 1} & 0 & $x^2$ + 1 & 1 & $x^2$ & $x$ & $x^2$ + $x$ + 1 & $x$ + 1 & $x^2$ + $x$ \\
\textbf{$x^2$+$x$} & 0 & $x^2$ + $x$ & $x^2$ + $x$ + 1 & 1 & $x^2$ + 1 & $x$ + 1 & $x$ & $x^2$ \\
\textbf{$x^2$+$x$ + 1} & 0 & $x^2$ + $x$ + 1 & $x^2$ + 1 & $x$ & 1 & $x^2$ + $x$ & $x^2$ & $x$ + 1
\end{tabular}}
\end{center}
Since field relies on operations from the prime field we used, it satisfies all the conditions of the field. From the above table it is also visible which elements are inverses of which element.
\end{proof}



\noindent\rule{12cm}{0.4pt}\\
\noindent \textbf{Exercise 4.3} Find out whether or not there is a finite field having 27, 36, 51, and 2401 elements, respectively. Justify your answer.
\begin{proof}
While finite fields with 27 and 2401 elements exist, there are no finite fields with 36 or 51 elements. Reason for this lies in the definition of \textbf{Finite fields}\footnote{Finite Fields - https://mathworld.wolfram.com/FiniteField.html}. The order of a finite field is always a prime or a power of a prime. Since 27 and 2401 are powers of primes, 3 and 7 respectively ($3^3,\ 7^4$) we can say that fields with 27 and 2401 elements exist. On the other hand, both 36 and 51 are neither a prime nor powers of prime, so we can say that no such finite field with 36 or 51 elements exists.
\end{proof}
\end{document}