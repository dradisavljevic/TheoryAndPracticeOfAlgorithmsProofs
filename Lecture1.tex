\documentclass{article}

\usepackage{amsmath,amsthm,amssymb}

\newcommand{\Lim}[1]{\raisebox{0.5ex}{{$\displaystyle \lim_{#1}\;$}}}
\newcommand{\subtitle}[1]{%
  \posttitle{%
    \par\end{center}
    \begin{center}\LARGE#1\end{center}
    \vskip0.5em}%
}

\begin{document}
\subtitle{Lecture I}\\
\noindent \textbf{Exercise 1.1} Prove or disprove the following: $f(x) = o(g(n))$ if and only if $\Lim{n \to \infty} \frac{f(n)}{g(n)}=0$.
\begin{proof}
  Let $\Lim{n \to \infty} \frac{f(n)}{g(n)}=0$, meaning that as n increases towards infinity, $\frac{f(n)}{g(n)}$ approaches 0. This means that in order for this to be true, the following statement must be true $g(n) > f(n)$ for very large n, so that the fraction $\frac{f(n)}{g(n)}$ is sufficiently small. In other words, $g(n)$ grows faster than $f(n)$ which can be expressed as $f(x) = o(g(n))$. \\
  
  Now let $\Lim{n \to \infty} \frac{f(n)}{g(n)}\neq0$ meaning that as n increases limit of the fraction $\frac{f(n)}{g(n)}$ equals to either some constant $c$ that is not 0 or approaches infinity. In the first case, where limit yields a constant $c$, it can be written as both $\Lim{n \to \infty} \frac{f(n)}{g(n)}>0$ and $\Lim{n \to \infty} \frac{f(n)}{g(n)}<\infty$ meaning that $g(n)$ is both upper and lower bound of $f(n)$, or in other words an exact bound, which can be written as $f(x) = \Theta(g(n))$. Since $g(n)$ is an upper bound of $f(n)$, this means that $f(x) \neq o(g(n))$. In the case of the limit being equal to infinity for a very large n, the statement $g(n) < f(n)$ must hold true, in other words $f(x) = O(g(n))$, which further implies $f(x) \neq o(g(n))$. In both cases we have that if $\Lim{n \to \infty} \frac{f(n)}{g(n)}\neq0$ then also $f(x) \neq o(g(n))$. \\
  
  Considering both $\Lim{n \to \infty} \frac{f(n)}{g(n)}=0 \implies f(x) = o(g(n))$ and $\Lim{n \to \infty} \frac{f(n)}{g(n)}\neq0 \implies f(x) \neq o(g(n))$, we can say that $f(x) = o(g(n)) \iff \Lim{n \to \infty} \frac{f(n)}{g(n)}=0$ which was what needed to be proven.

\end{proof}
\noindent\rule{12cm}{0.4pt}\\
\noindent \textbf{Exercise 1.2} Let $\mathcal{M}_n$ = (M_n,+_n,$\mathrm{x}_n$,0_n,$\mathrm{I}_n$). Then we have $\mathcal{M}_n$ is a ring with identity element $\mathrm{I}_n$ if and only if $R$ is ring with 1.
\begin{proof}
  Let 
\end{proof}

\end{document}