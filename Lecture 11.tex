\documentclass[a4paper]{article}

\usepackage{amsmath,amsthm,amssymb, titling, tikz, mdframed}

\title{\vspace{-2cm}Lecture XI\vspace{-2cm}}
\date{}

\newmdtheoremenv{theorem}{Theorem}
\newmdtheoremenv{definition}{Definition}
\newmdtheoremenv{corollary}{Corollary}
\newmdtheoremenv{algorithm}{Algorithm}
\newmdtheoremenv{lemma}{Lemma}

\begin{document}
\maketitle
\section{Lecture Summary}
\begin{theorem}
Let $\mathbb{F}$ be any field. Then there is a probabilistic algorithm which on input any $\varepsilon > 0$ and any three $n \times n$ matrices $A$, $B$, and $C$ over $\mathbb{F}$, where $n \in \mathbb{N}$, tests with probability $1 - \varepsilon$ whether or not $AB = C$ and whose running time is $O(\lceil \log{\frac{1}{\varepsilon}} + 1 \rceil n^{2})$.
\end{theorem}


\begin{definition}
We define the Legendre symbol $\left( \frac{a}{p} \right)$ as follows:
\begin{center}
\[ 
\left( \frac{a}{p} \right) = \left\{
\begin{array}{ll}
      1, & if\ a\ is\ a\ quadratic\ residue\ modulo\ p\; \\
      -1, & otherwise.\\
\end{array} 
\right. 
\]
\end{center}
\end{definition}


\begin{theorem}
Let $p$ be an odd prime and let $g \in \mathbb{Z}_{p}^{*}$ be a generator for $\mathbb{Z}_{p}^{*}$. Then for all $a \in \mathbb{Z}_{p}^{*}$ we have: $a$ is a quadratic residue modulo $p$ if and only if $dlog_{g}a$ is even.
\end{theorem}


\begin{corollary}
Let $p$ be an odd prime. Then there are precisely $(p - 1)/2$ many quadratic residues and $(p - 1)/2$ many quadratic nonresidues in $\mathbb{Z}_{p}^{*}$.
\end{corollary}


\begin{corollary}
Let $p$ be an odd prime. Then $\left( \frac{ab}{p} \right) = \left( \frac{a}{p} \right) \left( \frac{b}{p} \right)$ for all $a, b \in \mathbb{Z}_{p}^{*}$.
\end{corollary}


\begin{theorem}
Let $p$ be an odd prime and let $g$ be a generator of $\mathbb{Z}_{p}^{*}$. Then we have $g^{(p - 1) / 2} \equiv -1$ mod $p$.
\end{theorem}


\begin{theorem}
\textbf{Euler's Criterion -} Let $p$ be an odd prime and let $a \in \mathbb{Z}_{p}^{*}$, then
$$a^{(p - 1) / 2} \equiv \left( \frac{a}{p} \right)\ mod\ p$$
\end{theorem}


\begin{definition}
\textbf{Jacobi Symbol -} Let $Q > 1$ be an odd number, and let $Q = p_{1} \cdot p_{2} \cdot \cdots \cdot p_{k}$, where $p_{i}$ is prime fro all $i = 1, \ldots, k$ (but not necessarily $p_{i} \neq p_{j}$ for $i \neq j$). Let $a \in \mathbb{Z}_{Q}^{*}$. The Jacobi symbol $\left( \frac{a}{Q} \right)$ is defined as follows:
$$\left( \frac{a}{Q} \right) = \left( \frac{a}{p_{1}} \right) \cdot \left( \frac{a}{p_{2}} \right) \cdot \cdots \cdot \left( \frac{a}{p_{k}} \right).$$
\end{definition}


\begin{theorem}
Testing primality can be done in one-sided error probabilistic polynomial time.
\end{theorem}


\begin{algorithm}
\textbf{Algorithm Solovay Strassen Primality Test -} \\
Input: An odd number $n \in \mathbb{N}$ \\
Method: (1) Choose at random a number $a \in \{1, \ldots, n - 1 \}$. \\
(2) Compute $d = gcd(a, n)$. If $d > 1$ then output composite, and stop. Otherwise goto (3). \\
(3) Compute the following quantities:
\begin{center}
$\delta = a^{(n - 1)/2}\ mod\ n$; \\
$\varepsilon = \left( \frac{a}{n} \right)$ (the Jacobi symbol).
\end{center}
\begin{tabular}{c c}
Output: & If $\delta \not \equiv \varepsilon$ mod $n$ then output composite, and stop. \\
& If $\delta \equiv \varepsilon$ mod $n$ then output possibly prime, and stop
\end{tabular}
\end{algorithm}


\begin{theorem}
\textbf{Law of Quadratic Reciprocity -} For all odd numbers $P, Q \in \mathbb{N}$ with $gcd(Q, P) = 1$ we have
$$\left( \frac{Q}{P} \right) = (-1)^{(P - 1)(Q - 1)/4} \left( \frac{P}{Q} \right)$$
\end{theorem}


\begin{theorem}
For all $a, b \in \mathbb{N}$ and all odd $Q \in \mathbb{N}$ we have: \\
(1) if $a \equiv b$ mod $Q$ then $\left( \frac{a}{Q} \right) = \left( \frac{b}{Q} \right)$, \\
(2) $\left( \frac{1}{Q} \right) = 1$, \\ 
(3) $\left( \frac{-1}{Q} \right) = (-1)^{(Q - 1) / 2}$, \\
(4) $\left( \frac{ab}{Q} \right) = \left( \frac{a}{Q} \right) \cdot \left( \frac{b}{Q} \right)$, \\
(5) $\left( \frac{a}{Q} \right) = (-1)^{(Q^{2} - 1)/8}$.
\end{theorem}


\begin{samepage}
\begin{corollary}
If we run the Solovay Strassen Primality Test algorithm $k$-times then
\begin{center}
$Pr \{ k$ successive runs output "possibly prime" $\} \leq \frac{1}{2^{k}}$
\end{center}
provided $n$ is composite.
\end{corollary}
\end{samepage}



\section{Exercises}
\noindent \textbf{Exercise 11.1}  Apply  \textbf{Theorems 6 and 7} to compute $\left(\frac{119}{291}\right)$.
\begin{proof}
Computing Jacobi Symbol $\left(\frac{119}{291}\right)$ follows:
\begin{center}
\begin{tabular}{ c c c c  }
$\left(\frac{119}{291}\right)$ & $=$ & $(-1)^{(291-1)(119-1)/4}\left(\frac{291}{119}\right)$ & (\textasteriskcentered\textbf{Theorem 6}\textasteriskcentered)\\
& $ = $ & $ - \left(\frac{291}{119}\right)$ & \\
& $ = $ & $ - \left(\frac{53}{119}\right)$ & (\textasteriskcentered\textbf{Theorem 7} (1)\textasteriskcentered)\\
& $ = $ & $ - \left(\frac{119}{53}\right) = - \left(\frac{13}{53}\right)$  &\\
& $ = $ & $ - \left(\frac{53}{13}\right) = - \left(\frac{1}{13}\right)$  &\\
& $ = $ & $ - 1$ & (\textasteriskcentered\textbf{Theorem 7} (2)\textasteriskcentered)\\
\end{tabular}
\end{center}
So Jacobi Symbol here equals to $-1$
\end{proof}



\noindent\rule{12cm}{0.4pt}\\
\noindent \textbf{Exercise 11.2} Prove the following: Let $p \in \mathbb{N}$ be an odd prime and let $a \in \mathbb{Z}^{*}_{p}$.\\
(1) $ x^{2} \equiv -1$ mod $p$ is solvable if and only if $p \equiv 1$ mod 4.\\
(2) If $p \equiv 3$ mod $4$ then either $a$ or $-a$ is a quadratic residue.\\
(3) If $p \equiv 1$ mod $4$ then either both $a$ and $-a$ are quadratic residues or both are quadratic nonresidues.
\begin{proof}
\textbf{( 1 )} What here needs proving is that -1 is quadratic residue modulo $p$ and for which $p$ this is true. For \textbf{Legendre Symbol} that equals 1 (\textbf{Definition 1}) this is true. If we look at \textbf{Theorem 7} (3) We have that:
\begin{center}
$\left(\frac{-1}{Q}\right) = (-1)^{(Q-1)/2}$
\end{center}
From here we have that unless $Q \equiv 1$ mod $4$ the exponent won't equal to a positive whole integer (in other words, this equation won't yield $1$ as a solution). Since $p$ is a prime this implies that congruence $x^{2} \equiv -1$ mod $p$ is solvable if and only if $p \equiv 1$ mod $4$, which is what needed proving. \\
\textbf{( 2 )} From \textbf{Theorem 6} we have that:
\begin{center}
$\left(\frac{Q}{P}\right) = (-1)^{(P-1)(Q-1)/4}\left(\frac{P}{Q}\right)$
\end{center}
After multiplying both sides with$\left(\frac{P}{Q}\right)$ (since Legendre symbol can be either 1 or -1, squaring it on the right hand side yields one, and therefore removes it from the equation) we get:
\begin{center}
$\left(\frac{P}{Q}\right)\left(\frac{Q}{P}\right) = (-1)^{(P-1)(Q-1)/4}$
\end{center}
Now from solution in ( 1 ) we have that if $p \equiv 3$ mod $4$ then $-1$ is a nonresidue and $p \equiv 1$ mod $4$ then $-1$ is a residue. This implies that in the case of $p \equiv 3$ mod $4$, negative of a residue is a nonresidue and negative of a nonresidue is residue ($1$ is negative of $-1$) which implies that if $p \equiv 3$ mod $4$ either $a$ or $-a$ are residue.
\textbf{( 3 )} much like with the second claim, if we consider all the information we stated we have that $p \equiv 1$ mod $4$ implies that negative of a residue is also a residue, meaning that in this case if $a$ is residue, $-a$ is also a residue, while in the case $a$ is a nonresidue, then $-a$ is also a nonresidue.
\end{proof}



\noindent\rule{12cm}{0.4pt}\\
\noindent \textbf{Exercise 11.3} Apply the Algorithm PTSS (Solovay Strassen Primality Test) to check whether or not 11981 and 69147, respectively, is prime. Interpret the results obtained.
\begin{proof}
We start by checking for 11981. First we will select a number smaller than 11981, say 7, and calculate gcd(7,11981). Since it is 1, we then calculate $7^{(11981-1)/2}$ mod $11981$ and Jacobi symbol $\left(\frac{7}{111981}\right)$, and we get that both are equal. So for 7, it seems that $11981$ is possibly prime. We repeat the test for 100, 751 and 13, and get the same results where Jacobi symbol matches Euler's criterion. So 11981 is possibly prime.\\
For number 69147 it is obvious to the human eye that it is not prime, as the sum of its digits yields a number divisible by 3, so we will try applying the algorithm for $a=3$. Immediately on the second step we have that gcd(3,69147) is not 1, and we can say that the number is composite.
\end{proof}
\end{document}