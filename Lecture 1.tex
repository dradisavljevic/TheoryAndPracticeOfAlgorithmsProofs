\documentclass{article}

\usepackage{amsmath,amsthm,amssymb}

\newcommand{\Lim}[1]{\raisebox{0.5ex}{{$\displaystyle \lim_{#1}\;$}}}
\newcommand{\subtitle}[1]{%
  \posttitle{%
    \par\end{center}
    \begin{center}\LARGE#1\end{center}
    \vskip0.5em}%
}

\begin{document}
\subtitle{Lecture I}
\noindent \textbf{Exercise 1.1} Prove or disprove the following: $f(x) = o(g(n))$ if and only if $\Lim{n \to \infty} \frac{f(n)}{g(n)}=0$.
\begin{proof}
  Let $\Lim{n \to \infty} \frac{f(n)}{g(n)}=0$, meaning that as n increases towards infinity, $\frac{f(n)}{g(n)}$ approaches 0. This means that in order for this to be true, the following statement must be true $g(n) > f(n)$ for very large n, so that the fraction $\frac{f(n)}{g(n)}$ is sufficiently small. In other words, $g(n)$ grows faster than $f(n)$ which can be expressed as $f(x) = o(g(n))$. \\
  
  Now let $\Lim{n \to \infty} \frac{f(n)}{g(n)}\neq0$ meaning that as n increases limit of the fraction $\frac{f(n)}{g(n)}$ equals to either some constant $c$ that is not 0 or approaches infinity. In the first case, where limit yields a constant $c$, it can be written as both $\Lim{n \to \infty} \frac{f(n)}{g(n)}>0$ and $\Lim{n \to \infty} \frac{f(n)}{g(n)}<\infty$ meaning that $g(n)$ is both upper and lower bound of $f(n)$, or in other words an exact bound, which can be written as $f(x) = \Theta(g(n))$. Since $g(n)$ is an upper bound of $f(n)$, this means that $f(x) \neq o(g(n))$. In the case of the limit being equal to infinity for a very large n, the statement $g(n) < f(n)$ must hold true, in other words $f(x) = O(g(n))$, which further implies $f(x) \neq o(g(n))$. In both cases we have that if $\Lim{n \to \infty} \frac{f(n)}{g(n)}\neq0$ then also $f(x) \neq o(g(n))$. \\
  
  Considering both $\Lim{n \to \infty} \frac{f(n)}{g(n)}=0 \implies f(x) = o(g(n))$ and $\Lim{n \to \infty} \frac{f(n)}{g(n)}\neq0 \implies f(x) \neq o(g(n))$, we can say that $f(x) = o(g(n)) \iff \Lim{n \to \infty} \frac{f(n)}{g(n)}=0$ which is what was needed to be proven.

\end{proof}



\noindent\rule{12cm}{0.4pt}\\
\noindent \textbf{Exercise 1.2} Let $\mathcal{M}_n$ = ($M_n$,$+_n$,$\mathrm{x}_n$,$0_n$,$\mathrm{I}_n$). Then we have $\mathcal{M}_n$ is a ring with identity element $\mathrm{I}_n$ if and only if $\mathcal{R}$ is ring with 1.
\begin{proof}
  Let $\mathcal{R}$ = ($R$,$\circ$,$\mathcal{*}$,0,1) be a ring with 1. This means that $\mathcal{R}$ has following properties:\\
  \begin{itemize}
  \item Both $\circ$ and $\mathcal{*}$ operations are associative and closed over $R$.
  \item $\circ$ (Additive operation) is commutative.
  \item $\mathcal{*}$ (Multiplicative operation) is distributive with respect to $\circ$.
  \item 0 is neutral element with respect to $\circ$.
  \item 1 is identity element with respect to $\mathcal{*}$.
  \item For every $a$, where $a \in R$, there is an additive inverse $\neg a$ such that $a\circ\neg a=0$.
  \end{itemize}
  $M_n$ is defined as set of $ n\ \mathrm{x}\ n$ matrices over $R$ with $+_n$ and $\mathrm{x}_n$ being square matrix equivalent of additive and multiplicative operations, respectively, present in ring $\mathcal{R}$. Let matrices $\mathcal{A}$ and $\mathcal{B}$ be members of $M_n$ such that
  $\mathcal{A}= \begin{bmatrix}
 	a & b\\
 	c & d
 \end{bmatrix}$, $\mathcal{B}= \begin{bmatrix}
 	e & f\\
 	g & h
 \end{bmatrix}$ and $\mathcal{C}= \begin{bmatrix}
 	i & j\\
 	k & l
 \end{bmatrix}$. Applying the additive and multiplicative operations between these matrices we have: \\\\
 $\begin{pmatrix} \begin{bmatrix}
 	a & b\\
 	c & d
 \end{bmatrix}$ $+_n$ $\begin{bmatrix}
 	e & f\\
 	g & h
 \end{bmatrix} \end{pmatrix}$ $+_n$ $\begin{bmatrix}
 	i & j\\
 	k & l
 \end{bmatrix}$ = $\begin{bmatrix}
 	a\circ e\circ i & b\circ f\circ j\\
 	c\circ g\circ k & d\circ h\circ l
 \end{bmatrix}$ \\\\\\
 $ \begin{bmatrix}
 	a & b\\
 	c & d
 \end{bmatrix}$ $+_n$ $\begin{pmatrix} \begin{bmatrix}
 	e & f\\
 	g & h
 \end{bmatrix}$ $+_n$ $\begin{bmatrix}
 	i & j\\
 	k & l
 \end{bmatrix} \end{pmatrix}$ = $\begin{bmatrix}
 	a\circ e\circ i & b\circ f\circ j\\
 	c\circ g\circ k & d\circ h\circ l
 \end{bmatrix}$ \\\\\\
  $\begin{bmatrix}
 	a & b\\
 	c & d
 \end{bmatrix}$ $\mathrm{x}_n$ $\begin{bmatrix}
 	e & f\\
 	g & h
 \end{bmatrix}$ = $\begin{bmatrix}
 	(a\mathcal{*} e) \circ (b\mathcal{*} g) & (a\mathcal{*} f) \circ (b\mathcal{*} h)\\
 	(c\mathcal{*} e) \circ (d\mathcal{*} g) & (c\mathcal{*} f) \circ (d\mathcal{*} h)
 \end{bmatrix}$\\\\\\
  $\begin{bmatrix}
 	a & b\\
 	c & d
 \end{bmatrix}$ $+_n$ $\begin{bmatrix}
 	0 & 0\\
 	0 & 0
 \end{bmatrix}$ = $\begin{bmatrix}
 	a \circ 0 & b \circ 0\\
 	c \circ 0 & d \circ 0
 \end{bmatrix}$\\\\\\
  $\begin{bmatrix}
 	a & b\\
 	c & d
 \end{bmatrix}$ $\mathrm{x}_n$ $\begin{bmatrix}
 	1 & 0\\
 	0 & 1
 \end{bmatrix}$ = $\begin{bmatrix}
 	(a\mathcal{*} 1) \circ (b\mathcal{*} 0) & (a\mathcal{*} 0) \circ (b\mathcal{*} 1)\\
 	(c\mathcal{*} 1) \circ (d\mathcal{*} 0) & (c\mathcal{*} 0) \circ (d\mathcal{*} 1)
 \end{bmatrix}$\\\\\\
From this it can be seen that additive operation is both closed and associative. Additionally it can be seen that additive operation is commutative if the $\circ$ operation is commutative. Due to the fact that multiplication of matrices is associative, it can be deduced that $\mathrm{x}_n$ is associative if both $\circ $ and $\mathcal{*}$ are associative.\\
Taking into consideration that matrix multiplication is distributive in respect to matrix addition we have that $\mathrm{x}_n$ is distributive in respect to $+_n$ if $\mathcal{*}$ is in respect to $\circ $. \\
Additionally, we can see from the above equations that $0_n$ is a neutral element with respect to $+_n$ if 0 is neutral element with respect to $\circ$, while $\mathrm{I}_n$ is identity element with respect to $\mathrm{x}_n$ if 1 is identity element with respect to $\mathcal{*}$.\\
Lastly, if we consider the example of matrix addition above it can be seen that in order for additive inverse to exist, all that is necessary is for the two matrices $\mathcal{A}$ and $\mathcal{B}$ one of them to contain negative elements of the other ones, which relies on elements from $R$ having an additive inverse.\\
Thus we can conclude that if $\mathcal{R}$ is a ring with identity, $\mathcal{M}$ also is.\\
The opposite direction, that when $\mathcal{R}$ is not a ring with identity $\mathcal{M}$ is easily noticed, since all the operations in $\mathcal{M}$ rely on properties of operations in  $\mathcal{R}$ in order to fulfill conditions of a ring.
\end{proof}



\noindent\rule{12cm}{0.4pt}\\
\noindent \textbf{Exercise 1.3} Show that the definition of $+$ and $\cdot$ over $\mathbb{Z}_m$ are independent of the choice of the representation.
\begin{proof}
Sadly I have no idea how to prove this at current point.
\end{proof}



\noindent\rule{12cm}{0.4pt}\\
\noindent \textbf{Exercise 1.4} Prove or disprove: Let $(G, \circ)$ be any Abelian group, let $a \in G$, and let $n, m \in \mathbb{N}$. If $a$ has the order $n$ then $a^m=e$ if and only if $m \equiv 0$ mod $n$.
\begin{proof}
Suppose that $a$ has the order $n$. That would mean that $n$ is a smallest positive integer for which the statement $a^n=e$ holds true. Next let's suppose that $a^m=e$. This would mean that $a^n \equiv a^m$ mod n, which when applying \textbf{Theorem 1.5} would mean that $\prod_{i=1}^{m} a \equiv \prod_{i=1}^{n} a$ mod n, so $m$ would have to be divisible by $n$, or in other words $m \equiv 0$ mod $n$.\\
Now let's suppose that $m \equiv 0$ mod $n$ and $a^m=e$, this would mean that $m$ is either order of the group $G$ or a number larger than the one that is order. Next we have two cases $m=n$ and $m \neq n$. In first case, $n=m$ is order of $G$, and the exercises has been proven.\\
The second case is such that we can rewrite $m$ as $m=c\cdot n$ for some integer $c$, due to it being divisible by n. Considering $a^m=e$ we can rewrite it as ${a^n}^c$ which in turn we can write as $a^n=e^{1/c}$ which is equal to writing $a^n=e$, meaning $n$ is order of $a$.
\end{proof}



\noindent\rule{12cm}{0.4pt}\\
\noindent \textbf{Exercise 1.5} Show that $\sum_{k=0}^{998} k^3$ is divisible by 999.
\begin{proof}
Since for $k=0$ $k3$ also equals 0, we can rewrite the following sum as $\sum_{k=1}^{998} k^3$. Now, according to \textbf{Faulhaber's Formula}\footnote{Faulhaber's Formula - https://mathworld.wolfram.com/FaulhabersFormula.html} we can express $\sum_{k=1}^{998} k^3$ as $\frac{998^2 999^2}{4}$ due to $\sum_{k=1}^{n} k^3=\left( \frac{n(n+1)}{2}\right)^2$. From there we see that this equals $499^2 \cdot 999^2$ which is obviously divisible by 999.
\end{proof}




\noindent\rule{12cm}{0.4pt}\\
\noindent \textbf{Exercise 1.6} Show that every integer written in decimal representation is divisible by 3 if and only if the sum of its digits is divisible by 3.
\begin{proof}
Suppose we have a number $X$ with $n$ digits, we can write it as $X=x_n x_{n-1}x_{n-2}...x_2x_1x_0$ which we can then write as \\\\
 $X=x_0+x_1 \cdot 10 + x_2 \cdot 100 ... + x_{n-1} \cdot 10^{n-1} + x_n \cdot 10^n$ \\\\
 Which can further be rewritten as: \\\\
 $X = x_0 + x_1 \cdot (9 + 1) + x_2 \cdot (99 + 1) + x_3 \cdot (999 + 1)...$\\\\
 Or differently: \\\\
 $X = 9x_1 + 99x_2 + 999x_3 + ... + (x_0 + x_1 + x_2 + x_3 + ... + x_n)$ \\\\
 It is easy to notice that all the coefficients in the polynomial outside of parenthesis are multiples of 9 and thus divisible by 3, the only thing that is left to check for divisibility is polynomial within the parenthesis: \\\\
 $ x_0 + x_1 + x_2 + x_3 + ... x_{n-2} + x_{n-1} + x_n$ \\\\
 Which is coincidentally a sum of all $n$ digits of the number $X$. So, $X$ is divisible by 3 if and only if a sum of its n digits is divisible by 3. 
\end{proof}



\noindent\rule{12cm}{0.4pt}\\
\noindent \textbf{Exercise 1.7} Derive a criterion for the divisibility by 3 for the integers written in binary representation.
\begin{proof}
In order to write criterion for divisibility we will first write some numbers that are divisible by 3 as binary:\\
If we take $2^n$ values, we know that $2^0 = 1$ which is congruent 1 mod 3, and $2^1 = 2$ which is congruent 2 mod 3, or differently -1 mod 3. If we write binary numbers so that digit in even position multiplies -1 and the one in odd position multiplies 1, if we add them together we will get a number which, if divisible by 3, tells us that the binary number is divisible by 3. For example if we take number 309, we can write it as:\\
\begin{center}
$1 \cdot 1 + 0 \cdot (-1) + 0 \cdot 1 + 1 \cdot (-1) + 1 \cdot 1 + 0 \cdot (-1) + 1  \cdot 1 + 0 \cdot (-1) + 1 \cdot 1$ = 3
\end{center} 
Which is divisible by 3, thus 309 is divisible by 3. We have a table of first 30 numbers out of which 10 are divisible by 3: \\
\begin{center}
\begin{tabular}{ c c c c c  }
Decimal & Binary & Number of 1s in odd position & Number of 1s in even position & Difference \\
 1 & $0001_2$ & 1 & 0 & 1 \\ 
 2 & $0010_2$ & 0 & 1 & -1 \\ 
 3 & $0011_2$ & 1 & 1 & 0 \\   
 4 & $0100_2$ & 1 & 0 & 1 \\ 
 5 & $0101_2$ & 2 & 0 & 2 \\ 
 6 & $0110_2$ & 1 & 1 & 0 \\ 
 7 & $0111_2$ & 2 & 1 & 1 \\ 
 8 & $1000_2$ & 0 & 1 & -1 \\ 
 9 & $1001_2$ & 1 & 1 & 0 \\ 
 10 & $1010_2$ & 0 & 2 & -2 \\ 
 11 & $1011_2$ & 1 & 2 & -1 \\ 
 12 & $1100_2$ & 1 & 1 & 0 \\ 
 13 & $1101_2$ & 2 & 1 & 1 \\ 
 14 & $1110_2$ & 1 & 2 & -1 \\ 
 15 & $1111_2$ & 2 & 2 & 0 \\ 
 16 & $0001\ 0000_2$ & 1 & 0 & 1 \\ 
 17 & $0001\ 0001_2$ & 2 & 0 & 2 \\ 
 18 & $0001\ 0010_2$ & 1 & 1 & 0 \\ 
 19 & $0001\ 0011_2$ & 2 & 1 & 1 \\ 
 20 & $0001\ 0100_2$ & 2 & 0 & 2 \\ 
 21 & $0001\ 0101_2$ & 3 & 0 & 3 \\ 
 22 & $0001\ 0110_2$ & 2 & 1 & 1 \\ 
 23 & $0001\ 0111_2$ & 3 & 1 & 2 \\ 
 24 & $0001\ 1000_2$ & 1 & 1 & 0 \\ 
 25 & $0001\ 1001_2$ & 2 & 1 & 1 \\ 
 26 & $0001\ 1010_2$ & 1 & 2 & -1 \\ 
 27 & $0001\ 1011_2$ & 2 & 2 & 0 \\ 
 28 & $0001\ 1100_2$ & 2 & 1 & 1 \\ 
 29 & $0001\ 1101_2$ & 3 & 1 & 2 \\ 
 30 & $0001\ 1110_2$ & 2 & 2 & 0
\end{tabular}
\end{center}
So mathematically we can write:
\begin{center}
	$\sum_{r=0}^{n} 2^{r}b_{r} \equiv \sum_{r=0}^{n} {(-1)}^{r}b_{r}$
\end{center}
Where $b_{r}$ the bits since $2 \equiv -1$ mod 3. Thus coming to a criterion for divisibility of a binary number by 3.
\end{proof}



\noindent\rule{12cm}{0.4pt}\\
\noindent \textbf{Exercise 1.8} Compute the last two digits of $7^{50}$.
\begin{proof}
	If we calculate the first five powers of 7 we see that:
\begin{center}
	$7^1 = 7$\\
	$7^2 = 49$\\
	$7^3 = 343$\\
	$7^4 = 2401$\\
	$7^5 = 16907$
\end{center}
Because $7^5$ ends with 07, we know that cyclicity will occur, meaning that the last two digits will start repeating after 4 steps. Meaning that since $50 \equiv 2$ mod 4, we know that the last two digits of $7^{50}$ are going to be the same as in $7^2$, which is 49.
\end{proof}



\noindent\rule{12cm}{0.4pt}\\
\noindent \textbf{Exercise 1.9} Show that $3^{22}-2^{20}$ is divisible by 7.
\begin{proof}
Knowing that $3^2 \equiv 2$ mod 7, we have: \\
\begin{center}
	$3^{22}-2^{20} = {3^{2}}^{11}-2^{20} \equiv 2^{11} - 2^{20} \equiv {-2}^{11}(2^9-1)$
\end{center}
Now, we have $2^3 \equiv 1$ mod 7, so this implies:
 \begin{center}
	${-2}^{11}(2^9-1) \equiv {-2}^{11}(1-1) = 0$
\end{center}
Looking at this, we can claim that $3^{22}-2^{20}$ is divisible by 7.
\end{proof}



\noindent\rule{12cm}{0.4pt}\\
\noindent \textbf{Exercise 1.10} Determine for which $n \in \mathbb{N}$ the number $891^n - 403^n$ is divisible by 61.
\begin{proof}
We begin by looking at $n$ values from 1 to 6, and taking two numbers $a$ and $b$ such that $a, b \in \mathbb{N}$: \\
$n=1$
\begin{center}
	$a^1 - b^1 = a - b$
\end{center}
$n=2$
\begin{center}
	$a^2 - b^2 = (a - b)(a + b)$
\end{center}
$n=3$
\begin{center}
	$a^3 - b^3 = (a - b)(a^2 + ab + b^2)$
\end{center}
$n=4$
\begin{center}
	$a^4 - b^4 = (a - b)(a + b)(x^2 + y^2)$
\end{center}
$n=5$
\begin{center}
	$a^5 - b^5 = (a - b)(a^4 + a^{3}b + a^{2}b^{2} + ab^{3} + b^{4})$
\end{center}
$n=6$
\begin{center}
	$a^6 - b^6 = (a - b)(a + b)(a^2 + ab + b^2)(a^2 - ab + b^2)$
\end{center}
In a more general form we have following formula if n is an odd number: 
\begin{center}
	$a^{2n+1} - b^{2n-1} = (a - b)\sum_{j=0}^{2n}a^{2n-j}b^j$
\end{center}
\noindent And if n is an even number:
\begin{center}
	$a^{2n} - b^{2n} = (a - b)(a + b)\sum_{j=0}^{n-1}a^{2(n-j-1)}b^{2j}$
\end{center}
What is common for both formulae is the $(a - b)$ part. In the case of numbers 891 and 403, we can calculate that $891-403=488$ which is $488 \equiv 0$ mod 61, or in other words divisible by 61. So, due to the $(891-403)$ being present in both formulae, whether $n$ is odd or even, we can conclude that $891^n - 403^n$ is divisible by 61 for any $n \in \mathbb{N}$.
\end{proof}


\end{document}