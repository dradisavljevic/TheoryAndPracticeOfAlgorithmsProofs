\documentclass[a4paper]{article}

\usepackage{amsmath,amsthm,amssymb, titling, mdframed}

\title{\vspace{-2cm}Lecture XIV\vspace{-2cm}}
\date{}

\newmdtheoremenv{theorem}{Theorem}
\newmdtheoremenv{definition}{Definition}
\newmdtheoremenv{corollary}{Corollary}
\newmdtheoremenv{algorithm}{Algorithm}
\newmdtheoremenv{lemma}{Lemma}

\begin{document}
\maketitle
\section{Lecture Summary}
Different attributes between distributed algorithms: \\
(1) The Interprocess Communication (IPC) Method \\
(2) The Timing Model \\
(3) The Failure Model \\
(4) The Problems Addressed \\
A process $i, i= 1, \ldots, |V|$ consists formally of the following components: \\
$states_{i}$, a possibly infinite set of states; \\
$start_{i}$, a non-empty set of start states which is a subset of $states_{i}$ \\
$msg_{i}$, a message-generation function (depending only on $states_{i}$ and the edges outgoing from node $i$), and \\
$trans_{i}$, a state transition function mapping $states_{i} \times M^{*} \cup \{ \lambda \}$ to $states_{i}$.
\begin{theorem}
Let $n \in \mathbb{N}, n > 1$, and let $A$ be a system of $n$ processes arranged in a bidirectional ring. If all processes in $A$ are identical, then $A$ cannot solve the leader election problem.
\end{theorem}


\begin{theorem}
There exists an algorithm LE solving the leader election problem in time $n$ and with communication complexity $O(n^{2})$ for every system $A$ of $n$ processes arranged in a ring, where $n \in \mathbb{N}, n > 1$, is not known to the algorithm. Moreover, LE uses only unidirectional communication.
\end{theorem}


\begin{theorem}
\textbf{Hirschberg and Sinclair Theorem -} There exists an algorithm HS solving the leader election problem in time $O(n)$ and with communication complexity $O(n \log{n})$ for every system $A$ of $n$ processes arranged in a ring, where $n \in \mathbb{N}, n > 1$, is not known to the algorithm.
\end{theorem}


\begin{samepage}
\begin{algorithm}
Each process $i, i = 1, \ldots, n$ works in phases defined as follows: \\
\textbf{Phase $l, l \geq 0$}. Process $i$ does the following: \\
Initialize $(a, b, c)$ by $(u_{i}, out, 2^{l})$. Send in each direction one token with $(u_{i}, out, 2^{l})$. \\
Set status = unknown. \\
Set $t = 0$ (this variable is later used to count the tokens coming back). \\
end Initialization \\
Let $v = (u_{j}, b, k)$ be any incoming message. Then, if $b = in$ check whether $u_{j} = u_{i}$. If it is, set $t := t + 1$. If $t = 2$, then stop Phase $l$ and start Phase $l + 1$. \\
If $u_{j} \neq u_{i}$, then send $v$ in clockwise direction if it came from the counter-clockwise neighbor, and in counter-clockwise direction otherwise \\
If $b = out$ do the following: \\
\indent if $u_{j} > u_{i}$ then set $v = (u_{j}, b, k - 1)$ provided $k > 1$, otherwise set $v = (u_{j}, in, 0)$ and send $v$ to the appropriate neighbor, \\
\indent if $u_{j} < u_{i}$ then do nothing \\
\indent if $u_{j} = u_{i}$ then set \textbf{status = leader}. \\
The correctness formally follows mutatis mutandis as the proof of \textbf{Theorem 2}.
\end{algorithm}
\end{samepage}
\end{document}