\documentclass[a4paper]{article}

\usepackage{amsmath, amsthm, amssymb, titling, graphicx, mdframed}

\newcommand{\Lim}[1]{\raisebox{0.5ex}{{$\displaystyle \lim_{#1}\;$}}}
\title{\vspace{-2cm}Lecture I\vspace{-2cm}}
\date{}

\newmdtheoremenv{theorem}{Theorem}
\newmdtheoremenv{definition}{Definition}
\newmdtheoremenv{corollary}{Corollary}
\newmdtheoremenv{algorithm}{Algorithm}
\newmdtheoremenv{lemma}{Lemma}

\begin{document}
\maketitle
\section{Lecture Summary}
\begin{definition}
Let $g$: $\mathbb{N} \to \mathbb{R}_{\geq 0}$ be any function. We define the following sets:\\
(1) $O(g(n)) =_{df} \{f | f : \mathbb{N} \to \mathbb{R}_{\geq 0}$ there are constants,$c, n_{0} > 0$ such that $0 \leq f(n) \leq cg(n)$ for all $n \geq n_{0} \}$,\\
(2) $\Omega(g(n)) =_{df} \{f | f : \mathbb{N} \to \mathbb{R}_{\geq 0}$ there are constants $c, n_{0} > 0$ such that $0 \leq cg(n) \leq f(n)$ for all $n \geq n_{0} \}$,\\
(3) $o(g(n)) =_{df} \{f | f : \mathbb{N} \to \mathbb{R}_{\geq 0}$, for every constant $c > 0$ there exists a constant $n_{0} > 0$ such that $0 \leq f(n) \leq cg(n)$ for all $n \geq n_{0} \}$,\\
(4) $\Theta(g(n)) =_{df} \{f | f : \mathbb{N} \to \mathbb{R}_{\geq 0}$ there are constants $c_{1}, c_{2}, n_{0} > 0$ such that $0 \leq c_{1}g(n) \leq f(n) \leq c_{2}g(n)$ for all $n \geq n_{0} \}$
\end{definition}
$o$ and $O$ represent \textbf{asymptotic upper bound}, $\Omega$ \textbf{asymptotic lower bound} and $\Theta$ represents \textbf{asymptotic tight bound}. To indicate that a function $f$ is a member of $O(g(n))$, we write $f(n) = O(g(n))$.


\begin{theorem}
\textbf{The Master Theorem -} Let $a, b, c \in \mathbb{N}^{+}$, where $n \geq 2$. Then the recursive equation
\begin{center}
\[ 
T(n)=\left\{
\begin{array}{ll}
      b, & if\ n=1; \\
      aT\frac({n}{c})+bn, & for\ all\ n > 1,\\
\end{array} 
\right. 
\]
\end{center}
where $n \in \mathbb{N}$ is a power of $c$, has the following solution:
\begin{center}
\[ 
T(n)=\left\{
\begin{array}{ll}
      O(n), & if\ a < c; \\
      O(n \log{n}), & for\ all\ a=c;\\
      O(n^{\log_{c}{a}}), & for\ all\ a > c. \\
\end{array} 
\right. 
\]
\end{center}
\end{theorem}


\begin{definition}
\textbf{Group -} Let $G \neq \emptyset$ be any set, and let $\circ$: $G \times G \to G$ be any binary operation. We call ($G, \circ$) a group if\\
(1) $(a \circ b) \circ c$ = $a \circ (b \circ c)$ for all $a, b, c \in G$ (i.e., the operator $\circ$ is associative).\\
(2) There exists a neutral element $e \in G$ such that $a \circ e = e \circ a = a$ for all $a \in G$.\\
(3) For every $a \in G$ there exists an inverse element $b \in G$ such that $a \circ b = b \circ a$ for all $a,b \in G$.\\
(4) A group is called Abelian group if $\circ$ is also commutative, i.e., $a \circ b  = b \circ a$ for all $a, b \in G$.\\
(5) A group is said to be finite if $|G|$ is finite.\\
(6) If ($G, \circ$) is a finite group then we call $|G|$ the order of ($G, \circ$).
\end{definition}
If ($G, \circ$) satisfies the first axiom, we call it a semi-group.


\begin{theorem}
Let ($G, \circ$) be any group.Then we have the following:\\
(1) The neutral element of ($G, \circ$) is uniquely determined.\\
(2) For every $a \in G$ the inverse element $b$ of $a$ is uniquely determined.
\end{theorem}


\begin{definition}
Let ($G, \circ$) be a group and let $S \subseteq G$ be non-empty. Then ($S, \circ$) is said to be a subgroup of ($G, \circ$) if\\
(1) $a \circ b \in S$ for all $a, b \in S$;\\
(2) for every $a \in S$ also the inverse $b$ of $a$ is in $S$. 
\end{definition}


\begin{definition}
A group ($G, \circ$) is said to be a cyclic group if there exists an element $a \in G$ such that $G = \{ a^{n} | n \in \mathbb{Z} \}$. We refer to the element $a$ as a \textbf{generator} of $G$.
\end{definition}


\begin{definition}
Let ($G, \circ$) be any group with neutral element $e$, and let $a \in G$. The least number $n \in \mathbb{N}^{+}$ such that $a^{n} = e$ is called order of $a$ provided such an $n$ exists. If $a^{n} \neq e$ for all $n \in \mathbb{N}^{+}$ then we define the order of $a$ to be $\infty$. We denote the order of $a$ by ord($a$).
\end{definition}


\begin{theorem}
Let ($G, \circ$) be a finite group and let ($H, \circ$) be any subgroup of ($G, \circ$). Then the order of ($H, \circ$) divides the order of ($G, \circ$).
\end{theorem}


\begin{corollary}
Let ($G, \circ$) be any finite group with neutral element $e$, and let $a \in G$ be any element such that $ord(a) \neq \infty$. Then ord($a$) divides $|G|$.
\end{corollary}


\begin{definition}
\textbf{Ring -} Let $R$ be any non-empty set containing at least two distinguished elements $0$ and $1$, and let $\circ$ and $\mathcal{*}$ be binary operations over $R$ (that is $\circ$: $R \times R \to R$ and $\mathcal{*}$: $R \times R \to R$). Then $\mathcal{R} =(R, \circ, \mathcal{*}, 0, 1)$ is a ring with identity element provided that for all $a,b,c \in R$ the following properties hold:\\
(1) $(a \circ b) \circ c = a \circ (b \circ c)$ and $(a \mathcal{*} b) \mathcal{*} c = a \mathcal{*} (b \mathcal{*} c)$;\\
(2) $(a \circ b) = (b \circ a)$ ($\circ$ is commutative);\\
(3) $(a \circ b) \mathcal{*} c = (a \circ c) \mathcal{*} (b \circ c)$ and $a \mathcal{*} (b \circ c) = (a \mathcal{*} c) \circ (b \mathcal{*} c)$\\
(4) $a \circ 0 = 0 \circ a = a$ ($0$ is the \textbf{neutral element} with respect to $\circ$);\\
(5) $a \mathcal{*} 1 = 1 \mathcal{*} a = a$ ($1$ is the \textbf{identity element} with respect to $\mathcal{*}$);\\
(6) for each $a \in R$ there exists an element $-a \in R$ such tat $a \circ (-a) = (-a) \circ a = 0$ ($-a$ is the \textbf{additive inverse} of $a$).
\end{definition}
$\mathcal{R} =(R, \circ, \mathcal{*}, 0, 1)$ is a ring if ($R, \circ$) is an Abelian group, ($R, \mathcal{*}$) is a semi-group, and the laws of distributivity are satisfied. If axiom 5 is satisfied, we call it a ring with identity element. If the multiplicative operation is commutative, then we call ring a \textbf{commutative ring}.


\begin{definition}
\textbf{Field -} Let $F \neq \emptyset$ be any set containing two distinguished elements $0$ and $1$, $0 \neq 1$, and let $\circ$, $\mathcal{*}$: $F \times F \to F$ be two binary operations. We call $\mathbb{F} = (F, \circ, \mathcal{*}$) a field if:\\
(1) ($F, \circ$) is an Abelian group (with neutral element $0$).\\
(2) ($F \backslash \{0\}, \mathcal{*}$) is a group (with neutral element $1$).\\
(3) The following distributive laws are satisfied:
\begin{center}
$a \mathcal{*} (b \circ c) = (a \mathcal{*} c) \circ (b \mathcal{*} c)$\\
$(a \circ b) \mathcal{*} c = (a \circ c) \mathcal{*} (b \circ c)$
\end{center}
(4) A field ($F, \circ, \mathcal{*}$) is said to be Abelian (or commutative) if $a \mathcal{*} b = b \mathcal{*} a$ for all $a, b \in F \backslash \{0\}$ holds.
\end{definition}
We refer to $0$ as neutral element and $1$ as the identity element. $\mathbb{F}$ is a finite field if $|F|$ is finite.


\begin{theorem}
Let($F, \circ, \mathcal{*}$) be any field. Then we have\\
(1) $a \mathcal{*} 0 = 0 \mathcal{*} a = 0$ for all $a \in F$;\\
(2) $a \mathcal{*} b = 0$ iff $a = 0$ or $b = 0$;\\
(3) for all $a, b \in F$, $a \neq 0$ there is precisely one $x \in F$ such that $a \mathcal{*} x = b$.
\end{theorem}


In equation $a = qm + r$, where $0 \leq r < m$ we call $q$ the \textbf{integer quotient} and $r$ the \textbf{remainder}.\\
We write $a \equiv b$ mod $m$ iff $m$ divides $a -b$ ($m|(a-b)$). We then say that \textbf{$a$ is congruent $b$ modulo $m$} and refer to $\equiv$ as congruence relation.
\begin{definition}
We define addition and multiplication of these equivalence classes by\\
\begin{center}
$[a] + [b] =_{df} [a+b]$ $\quad$ and\\
$[a] \cdot [b] =_{df} [a \cdot b]$. $\qquad$
\end{center}
\end{definition}
From here it is easy to see that ($\mathbb{Z}_{m}, +, \cdot$) constitutes a commutative ring with $1$.


\begin{theorem}
Let $m \in mathbb{N}^{+}$, let $a, b, c, d \in \mathbb{Z}$ be any integers such that $a \equiv b$ mod $m$ and $c \equiv d$ mod $m$, and let $n \in \mathbb{N}$. Then we have the following:\\
(1) $a + c \equiv b + d$ mod $m$;\\
(2) $a - c \equiv b - d$ mod $m$;\\
(3) $ac \equiv bd$ mod $m$;\\
(4) $a^{n} \equiv b^{n}$ mod $m$.
\end{theorem}



\begin{samepage}
\section{Exercises}
\noindent \textbf{Exercise 1.1} Prove or disprove the following: $f(x) = o(g(n))$ if and only if $\Lim{n \to \infty} \frac{f(n)}{g(n)}=0$.
\begin{proof}
Let $\Lim{n \to \infty} \frac{f(n)}{g(n)}=0$, meaning that as n increases towards infinity, $\frac{f(n)}{g(n)}$ approaches $0$. This means that in order for this to be true, the following statement must be true $g(n) > f(n)$ for very large n, so that the fraction $\frac{f(n)}{g(n)}$ is sufficiently small. In other words, $g(n)$ grows faster than $f(n)$ which can be expressed as $f(x) = o(g(n))$. \\
Now let $\Lim{n \to \infty} \frac{f(n)}{g(n)}\neq0$ meaning that as $n$ increases limit of the fraction $\frac{f(n)}{g(n)}$ equals to either some constant $c$ that is not $0$ or approaches infinity. In the first case, where limit yields a constant $c$, it can be written as both $\Lim{n \to \infty} \frac{f(n)}{g(n)}>0$ and $\Lim{n \to \infty} \frac{f(n)}{g(n)}<\infty$ meaning that $g(n)$ is both upper and lower bound of $f(n)$, or in other words an exact bound, which can be written as $f(x) = \Theta(g(n))$. Since $g(n)$ is an upper bound of $f(n)$, this means that $f(x) \neq o(g(n))$. In the case of the limit being equal to infinity for a very large n, the statement $g(n) < f(n)$ must hold true, in other words $f(x) = O(g(n))$, which further implies $f(x) \neq o(g(n))$. In both cases we have that if $\Lim{n \to \infty} \frac{f(n)}{g(n)}\neq0$ then also $f(x) \neq o(g(n))$. \\
Considering both $\Lim{n \to \infty} \frac{f(n)}{g(n)}=0 \implies f(x) = o(g(n))$ and $\Lim{n \to \infty} \frac{f(n)}{g(n)}\neq0 \implies f(x) \neq o(g(n))$, we can say that $f(x) = o(g(n)) \iff \Lim{n \to \infty} \frac{f(n)}{g(n)}=0$ which is what was needed to be proven.
\end{proof}
\end{samepage}



\noindent\rule{12cm}{0.4pt}\\
\noindent \textbf{Exercise 1.2} 
\begin{proof}
Let $\mathcal{M}_{n} = (M_{n}, +_{n}, \times_{n}, 0_{n}, I_{n})$. Then we have: $\mathcal{M}_{n}$ is a ring with identity element $I_{n}$ iff $\mathcal{R}$ is ring with 1.\\
Necessity: Let $\mathcal{R}$ be a ring with $1$. We have to show that $\mathcal{M}_{n}$ is a ring with identity. Let $A$ = ($a_{ij}$), $B$ = ($b_{ij}$), and let $C$ = ($c_{ij}$), where $i,j = 1, \ldots, n$. We start with the associativity of matrix addition. Using that addition in the ring $\mathcal{R}$ we directly obtain
\begin{center}
$(A +_{n} B) +_{n} C = ((a_{ij} + b_{ij}) + c_{ij}) = (a_{ij} + (b_{ij} + c_{ij})) = A +_{n} (B +_{n} C),$
\end{center}
and the associativity of matrix addition is shown.\\
Using the commutativity of the ring addition, we also have
\begin{center}
$A +_{n} B = (a_{ij} + b_{ij}) = (b_{ij} + a_{ij}) = B +_{n} A,$
\end{center}
and so matrix addition is also commutative.\\
Next, we take into account that $0$ is the neutral element with respect to addition in $\mathcal{R}$. Therefore, using the definition of $0_{n}$ we see that
\begin{center}
$A +_{n} 0_{n} = (a_{ij} + 0) = (0 + a_{ij}) = (a_{ij}) = A$.
\end{center}
Consequently, the matrix $0_{n}$ is the neutral element with respect to matrix addition.\\
We claim that $I_{n}$ = ($i_{ij}$) is the identity element with respect matrix multiplication. Using the definition of matrix multiplication and the identity matrix $I_{n}$ as well as the facts that $1$ is the identity in $\mathcal{R}$ and that $0 \cdot a = a \cdot 0 = 0$ for every $a \in R$, this can be seen as follows: For the element at position ($i, j$) we have
$$A \times_{n} I_{n} = \left( \sum_{k=1}^{n}{a_{ik}i_{kj}}\right) = (a_{ij}) = \left( \sum_{k=1}^{n}{i_{ik}a_{kj}}\right) = A.$$
Next we show that $-A =_{df} (-a_{ij})$ is the additive inverse of the matrix $A$. Using that $a + (-a) = (-a) +a = 0$ for every $a \in R$, and the definition of matrix addition, we have
$$A +_{n} (-A) = (a_{ij} + (-a_{ij})) = ((-a_{ij}) + a_{ij}) = 0_{n} .$$
We continue with the associativity of matrix multiplication. Here we use that the ring multiplication is associative and that the laws of distributivity hold in $\mathcal{R}$. Note that now $i, j, l = 1, \ldots , n$. We obtain for the element at position ($i, l$) of the matrix product the following:
$$A \times_{n} (B \times_{n} C) = \left(\sum_{j=1}^{n}{a_{ij}} \cdot \left( \sum_{k=1}^{n}{b_{jk}c_{kl}}\right)\right) = \left(\sum_{j=1}^{n}\sum_{k=1}^{n}{a_{ij} \cdot b_{jk} \cdot c_{kl}} \right)=$$\\ $$= \left(\sum_{k=1}^{n}\sum_{j=1}^{n}{a_{ij} \cdot b_{jk} \cdot c_{kl}}\right) = \left( \sum_{k=1}^{n}\left(\sum_{j=1}^{n}{a_{ij} \cdot b_{jk}} \right) \cdot c_{kl} \right) = (A \times_{n} B) \times_{n} C.$$
To finish the proof of the necessity we also have to show the laws of distributivity for the ring $\mathcal{M}_{n}$. These laws are a direct consequence of the corresponding laws in $\mathcal{R}$. So we show here just one.
$$(A +_{n} B) \times_{n} C = \left(\sum_{k=1}^{n}{(a_{ik}+b_{ik}) \cdot c_{kj}}\right) = \left(\sum_{k=1}^{n}{a_{ik}c_{kj}+b_{ik}c_{kj})}\right)=$$\\ $$= \left(\sum_{k=1}^{n}{a_{ik}c_{kj}}\right) + \left(\sum_{k=1}^{n}{b_{ik}c_{kj}}\right) = A \times_{n} C +_{n} B \times_{n} C,$$
and the necessity is shown.\\
Sufficiency: The sufficiency is trivial, since it suffices to consider $1 \times 1$ matrices.
\end{proof}



\noindent\rule{12cm}{0.4pt}\\
\noindent \textbf{Exercise 1.3} Show that the definition of $+$ and $\cdot$ over $\mathbb{Z}_m$ are independent of the choice of the representation.
\begin{proof}
Let $x$, $u$, $v$ and $y$ be integers where we have $x \equiv u$ mod $m$ and $y \equiv v$ mod $m$, then it is also true that $x -u$ and $y - v$ are divisible by $m$. If we put the two equations together we can also see that $(x + y) - (u + v)$ is then also divisible by $m$ and from there we have that $(x+y) \equiv (u+v)$ mod $m$. We also have that $xy - uv = (x-u)y + u(y-v)$ from where it follows that $xy-uv$ is divisible by $m$ and thus $xy \equiv uv$ mod $m$.\\
We conclude that there are well-defined operations of addition and multiplication on the set $\mathbb{Z}_{m}$ of congruence classes of integers modulo $m$: the sum of the congruence classes of integers $x$ and $y$ is the congruence class of $x+y$, and the product of these congruence classes is the congruence class of $xy$. If $x$ and $u$ belong to the same congruence class and if $y$ and $v$ belong to the same congruence class, then we have shown that $x+y$ and $u+v$ belong to the same congruence class; we have also shown that $xy$ and $uv$ belong to the same congruence class. Thus we can conclude that these operations of addition and multiplication on congruence classes do not depend on the choice of representatives of those congruence classes.
\end{proof}



\noindent\rule{12cm}{0.4pt}\\
\noindent \textbf{Exercise 1.4} Prove or disprove: Let $(G, \circ)$ be any Abelian group, let $a \in G$, and let $n, m \in \mathbb{N}$. If $a$ has the order $n$ then $a^m=e$ if and only if $m \equiv 0$ mod $n$.
\begin{proof}
Since $n$ is the order of $a$, by \textbf{Definition 5} we have $a^{n} = e$ and
\begin{center}
$a^{l} \neq e$ for all $l \in \{1, \ldots, n-1 \}$.
\end{center}
If $m=0$ then we have $a^{0}=e$ and $m\equiv 0$ mod $n$. Next suppose any $m \in \mathbb{N}^{+}$ such that $a^{m} = e$. Then using division with remainder, we know that there are $q,r \in \mathbb{N}$ such that $m = nq +r$, where $0 \leq r \leq n-1$. Consequently we obtain
\begin{center}
$a^{m} = a^{nq+r} = (a^{n})q \circ a^{r} = e^{q} \circ a^{r} = e \circ a^{r} = a^{r}$.
\end{center}
We distinguish the following cases:\\
Case 1. $r=0$\\
Then $a^{m} =e$ and $m \equiv 0$ mod $n$.\\
Case 2. $r \neq 0$\\
Then by (1) we have $a^{r} \neq e$ and $m \not\equiv 0$ mod $n$.\\
Consequently, $a^{m} = e$ is possible iff $r=0$, which is equivalent to $m \equiv 0$ mod $n$. Thus, the assertion is true.
\end{proof}



\noindent\rule{12cm}{0.4pt}\\
\noindent \textbf{Exercise 1.5} Show that $\sum_{k=0}^{998} k^3$ is divisible by $999$.
\begin{proof}
Since for $k=0$ $k3$ also equals $0$, we can rewrite the following sum as $\sum_{k=1}^{998} k^3$. Now, according to \textbf{Faulhaber's Formula}\footnote{Faulhaber's Formula - https://mathworld.wolfram.com/FaulhabersFormula.html} we can express $\sum_{k=1}^{998} k^3$ as $\frac{998^2 999^2}{4}$ due to $\sum_{k=1}^{n} k^3=\left( \frac{n(n+1)}{2}\right)^2$. From there we see that this equals $499^2 \cdot 999^2$ which is obviously divisible by $999$.
\end{proof}




\noindent\rule{12cm}{0.4pt}\\
\noindent \textbf{Exercise 1.6} Show that every integer written in decimal representation is divisible by $3$ if and only if the sum of its digits is divisible by $3$.
\begin{proof}
Since the divisibility by 3 is not affected by the sign, it suffices to consider $z= \sum_{i=0}^{n}{z_{i}10^{i}}$, where $z_{i} \in \{0,1,\ldots,0\}$ for all $i = 0, \ldots, n$. Then by the reflexivity of "$\equiv$" we have
\begin{equation}
z_{i} \equiv z_{i}\ mod\ 3\ for\ all\ i = 0, \ldots n.
\end{equation}
Clearly, $10 \equiv 1$ mod $3$ and thus, Property (4) of \textbf{Theorem 5} implies
\begin{equation}
10^{i} \equiv 1^{i} \equiv 1\ mod\ 3\ for\ all\ i = 0, \ldots n.
\end{equation}
Next we apply Property (1) of \textbf{Theorem 5} to (1) and (2) exactly $n$ many times and obtain $\sum_{i=0}^{n}{z_{i}10^{i}} \equiv \sum_{i=0}^{n}{z_{i}}$ mod $3$. Hence we have shown that every integer written in decimal notation is divisible by $3$ iff the sum of its digits is divisible by $3$. 
\end{proof}



\noindent\rule{12cm}{0.4pt}\\
\noindent \textbf{Exercise 1.7} Derive a criterion for the divisibility by $3$ for the integers written in binary representation.
\begin{proof}
We use the same ideas as in the proof of \textbf{Exercise 1.6}. Let $x$ be given in binary notation. i.e., $z=\sum_{i=0}^{n}{z_{i}2^{i}}$, where $z_{i} \in \{ 0,1 \}$ for all $i = 0, \ldots, n$. Again, we have
\begin{equation}
z_{i} \equiv z_{i}\ mod\ 3
\end{equation}
as before, but (2) translates into
\begin{equation}
2^{i} \equiv (-1)^{i}\ mod\ 3\ for\ all\ i=0, \ldots, n.
\end{equation}
Thus, now we obtain $\sum_{i=0}^{n}{z_{i}2^{i}} \equiv \sum_{i=0}^{n}{(-1)^{n}z_{i}}$ mod $3$.
\end{proof}



\noindent\rule{12cm}{0.4pt}\\
\noindent \textbf{Exercise 1.8} Compute the last two digits of $7^{50}$.
\begin{proof}
We have to compute $7^{50}$ mod $100$, since this will give us the last two digits. We observe that $7^{4} \equiv 2401 \equiv 1$ mod $100$. Hence, Property (4) of \textbf{Theorem 5} implies that
\begin{center}
$(7^{4})^{12} \equiv 1^{12} \equiv 1$ mod $100$.
\end{center}
Since $4 \cdot 12 = 48$, we conclude by Property (3) of \textbf{Theorem 5} that
\begin{center}
$7^{50} \equiv 7^{48} \cdot 7^{2} \equiv 1 \cdot 49 \equiv 49$ mod $100$.
\end{center}
Thus the last two digits of $7^{50}$ are $49$.
\end{proof}



\noindent\rule{12cm}{0.4pt}\\
\noindent \textbf{Exercise 1.9} Show that $3^{22}-2^{20}$ is divisible by $7$.
\begin{proof}
We observe that $2^{3} \equiv 1$ mod $7$ and thus $2^{18} \equiv 1$ mod $7$ (from \textbf{Theorem 5}). Therefore, we have $2^{20} \equiv 4$ mod $7$.\\
Next $3^{2} \equiv 2$ mod $7$ and so $3^{6} \equiv 2^{3} \equiv 1$ mod $7$. Hence, we obtain $3^{18} \equiv 1$ mod $7$. Finally, since $3^{4} \equiv 4$ mod $7$, we thus have $3^{22} \equiv 4$ mod $7$. Consequently, we directly see that $3^{22} - 2^{20} \equiv 4 -4 \equiv 0$ mod $7$.
\end{proof}



\noindent\rule{12cm}{0.4pt}\\
\noindent \textbf{Exercise 1.10} Determine for which $n \in \mathbb{N}$ the number $891^n - 403^n$ is divisible by $61$.
\begin{proof}
Since $891 \equiv 37$ mod $61$ and $403 \equiv 37$ mod 61, by transitivity of the congruence relation we have $891 \equiv 403$ mod $61$. By Property (4) of \textbf{Theorem 5} we conclude that $891^{n}-403^{n}$ is divisible by $61$ for every $n \in \mathbb{N}$.
\end{proof}


\end{document}