\documentclass[a4paper]{article}

\usepackage{amsmath,amsthm,amssymb, titling}

\title{\vspace{-2cm}Lecture V\vspace{-2cm}}
\date{}

\begin{document}
\maketitle
\noindent \textbf{Exercise 5.1}  Let $p$ be a prime number. Then $\mathbb{Z}^{*}_{p^2}$ is cyclic.
\begin{proof}
According to the definition of \textbf{Finite Fields}\footnote{Finite Fields - https://mathworld.wolfram.com/FiniteField.html} there exists exactly one, up to an isomorphism, finite field with order equal to the power of prime. Since $p$ is a prime, this is true for $\mathbb{Z}_{p^2}$. Now we will prove that for a finite field a multiplicative group of it is cyclic.\\
Lets mark order of $\mathbb{Z}_{p^2}^{*}$ as $m$, and let's say there exists an element $w \in \mathbb{Z}_{p^2}^{*}$ whose order is maximal possible order of elements under multiplication in $\mathbb{Z}_{p^2}^{*}$, and lets mark it with $k$. According to \textbf{Lagrange's Theorem} (\textbf{Theorem 1.3}, slide 43) $k \mid m$ so we can say that $k \leq m$.\\
Now let's choose any other element $v \in \mathbb{Z}_{p^2}^{*}$, and let's mark its order as $r$. As a consequence of \textbf{Lagrange's Theorem}, there has to be an element in $\mathbb{Z}_{p^2}^{*}$ with the order equal to $lcm(k,r)$, however, since $k$ is maximal order, we will have that $lcm(k,r)=k$, which means $r \mid k$. Since order of $v$ is $r$ and $r \mid k$ we have that $v^{k} = 1$ ($k$ can be represented as a multiple of $r$, and since $v$ to the power of $r$ yields 1, 1 to the power of any multiple would still be 1). Since $v$ is arbitrary, we have that every element in $\mathbb{Z}_{p^2}^{*}$ is a zero of the polynomial $x^k -1$, that is $x^k - 1$ has $m$ roots in $\mathbb{Z}_{p^2}$. According to \textbf{Theorem 3.5} from slide 25, any polynomial has at most $d$ zeros in a a field, meaning that $m \leq k$. Since we have both $m \leq k$ and $k \leq m$, we can say that $m = k$.\\
Since element $w$ has order equal to order of$\mathbb{Z}_{p^2}^{*}$, it is a generator, and we have that $\mathbb{Z}_{p^2}^{*}$ is cyclic.
\end{proof}



\noindent\rule{12cm}{0.4pt}\\
\noindent \textbf{Exercise 5.2} Every Carmichael number is the product of at least 3 distinct primes.
\begin{proof}
Since Carmichael numbers are not prime, let's try to prove that there is a Carmichael number that has a form of $a=pq$, where both $p$ and $q$ are distinct prime numbers, or in other words, show that Carmichael number is product of at least two distinct primes. According to the first assertion of \textbf{Theorem 5.9}, slide 50, Carmichael numbers are square-free, so $p$ and $q$ have to be distinct. Also, let's assume that $p$ is a larger prime of the two, in other words $p > q$.\\
According to the second assertion of the \textbf{Theorem 5.9} we have that $a$ is a Carmichael number iff ($p$-1) divides ($a$-1) for ever $p$ that divides $a$. For this to be true, since $a = pq$, we have:
\begin{center}
$(p-1) \mid (pq - 1)$
\end{center}
Since prime $p$ divides $pq$. From here we have that there is some integer $k$ such that:
\begin{center}
$(p-1)k = pq - 1$
\end{center}
And if we multiply $(p-1)$ with a prime $q$, we would get:
\begin{center}
$(p-1)q = pq - q$
\end{center}
If we subtract the last two equations we will get:
\begin{center}
$(p-1)(k-q) = q - 1$
\end{center}
Which would mean that $(p-1) \mid (q-1)$, which is impossible since $p$ is a larger prime of the two distinct primes. This means that there is no Carmichael number that is product of two distinct primes, meaning that Carmichael number is product of at least three distinct primes.
\end{proof}



\noindent\rule{12cm}{0.4pt}\\
\noindent \textbf{Exercise 5.3} Prove equation (2) in \textbf{Example 5.1}.
\begin{proof}
The equations (2) from the said example is as follows:
\begin{center}
$x(x + 1)(x^3 + x + 1)(x^3 + x^2 + 1) = x^8 - x$
\end{center}
over $\mathbb{Z}_{2}$. Since in $\mathbb{Z}_{2}$ 1+1 = 0, we have that:
\begin{center}
$-1 = 1$\\
$2x = 0$
\end{center}
With that in mind we can expand the left hand side of the equation:
\begin{center}
$x(x + 1)(x^3 + x + 1)(x^3 + x^2 + 1) = x^8 - x$\\
$(x^2 + x)(x^3 + x + 1)(x^3 + x^2 + 1) = x^8 - x$\\
$(x^5 + x^3 + x^2 + x^4 + x^2 + x)(x^3 + x^2 + 1) = x^8 - x$
\end{center}
Since we have that $x^2 + x^2 = 0$ over $\mathbb{Z}_{2}$ this is equal to:
\begin{center}
$(x^5 + x^4 + x^3 + x)(x^3 + x^2 + 1) = x^8 - x$\\
$(x^8 + x^7 + x^5 + x^7 + x^6 + x^4 + x^6 + x^5 + x^3 + x^4 + x^3 + x) = x^8 - x$
\end{center}
Again, considering that $2x^7=0$, $2x^4 = 0$, $2x^3 = 0$, $2x^6 = 0$ and $2x^5=0$ over $\mathbb{Z}_{2}$ this is same as:
\begin{center}
$x^8 + x = x^8 - x$
\end{center}
And since in $\mathbb{Z}_{2}$ -1 = 1, the above equation is true, which is what needed proving.
\end{proof}



\noindent\rule{12cm}{0.4pt}\\
\noindent \textbf{Exercise 5.4} Calculate $5^{71}$ mod 31 and provide your solution in the way of the example given at page 39.
\begin{proof}
We start by representing 71 in binary form. Since 71 = 64 + 4 + 2 + 1, we know that in binary system it looks like $0100\ 0111_{2}$. So after taking that $y_0 = 1$ we can calculate:
\begin{center}
$y_1 \equiv 5^1 \equiv 5$ mod 31\\
$y_2 \equiv 5^2 \equiv 25 \equiv -6$ mod 31\\
$y_3 \equiv 5^3 \equiv 125 \equiv 26 \equiv -5$ mod 31\\
$y_4 \equiv 5^4 \equiv (-6)^2 \equiv 36 \equiv 5$ mod 31\\
$y_5 \equiv 5^5 \equiv 3125 \equiv 25 \equiv -6$ mod 31\\
$y_6 \equiv 5^6 \equiv (-5)^2 \equiv 25 \equiv -6$ mod 31\\
$y_7 \equiv 5^7 \equiv 78125 \equiv 5$ mod 31\\
\end{center}
So finally we have that:
\begin{center}
$5^{71} \equiv 5^{64+4+2+1} \equiv 5^{64}5^{4}5^{2}5^{1} \equiv 5 \cdot 5 \cdot (-6) \cdot 5 \equiv 5 \cdot 5 \cdot 25 \cdot 5 \equiv 5^5$
\end{center}
So we have that $5^71 \equiv 5^5 \equiv 25$ mod 31.\\
This could also be done by noticing that 31 is a prime number, which would mean that $5^{30}$ is 1, and $5^{60}$ is $1^2$ mod 31 (\textbf{Fermat's Little Theorem}). So we would look at $5^{71}$ as $5^{11}$ mod 31.
\end{proof}



\noindent\rule{12cm}{0.4pt}\\
\noindent \textbf{Exercise 5.5} Prove Assertion (2) of \textbf{Theorem 5.8}. The assertion reads:
\begin{center}
If $n$ is pseudo-prime to the bases $b_1$ and $b_2$ such that gcd($b_1$, $n$) = 1 and gcd($b_2$, $n$) = 1, then $n$ is also pseudo-prime to the bases $b_{1}b_{2}$, $b_{1}b_{2}^{-1}$, and $b_{1}^{-1}b_{2}$.
\end{center}
\begin{proof}
Since $n$ is a composite number for which $b_{1}^{n-1} \equiv 1$ mod $n$ and $b_{2}^{n-1} \equiv 1$ mod $n$, or in other words $b_{1}^n \equiv b_{1}$ mod $n$ and $b_{2}^n \equiv b_{2}$ mod $n$, this implies that $(b_{1}b_{2})^n = b_{1}^{n}b_{2}^{n} \equiv b_{1}b_{2}$ mod $n$, which is same as $(b_{1}b_{2})^{n-1} \equiv 1$ mod $n$, meaning $n$ is also pseudo-prime to the base $b_{1}b_{2}$.\\
Having $n$ be pseudo-prime to the bases $b_1$ and $b_2$ implies that $b_1^{-n} = {(b_1^{n})}^{-1} \equiv b_1^{-1}$ mod $n$ and $b_2^{-n} = {(b_2^{n})}^{-1} \equiv b_2^{-1}$ mod $n$ respectively, or that $n$ is pseudo-prime to bases $b_{1}^{-1}$ and $b_{2}^{-1}$ also. From this and the proof that $n$ is pseudo-prime for the base $b_{1}b_{2}$ we have that it is also pseudo-prime for base $b_{1}b_{2}^{-1}$, and $b_{1}^{-1}b_{2}$, which is what needed proof.
\end{proof}
\end{document}