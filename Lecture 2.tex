\documentclass[a4paper]{article}

\usepackage{amsmath,amsthm,amssymb, titling}

\title{\vspace{-2cm}Lecture II\vspace{-2cm}}
\date{}

\newmdtheoremenv{theorem}{Theorem}
\newmdtheoremenv{definition}{Definition}
\newmdtheoremenv{corollary}{Corollary}

\begin{document}
\maketitle
\section{Lecture Summary}
\begin{theorem}
The addition of two members $a$ and $b$ each having at most $n$ bits can be performed in time $O(n)$.
\end{theorem}



\section{Exercises}
\noindent \textbf{Exercise 2.1} Let $n \in \mathbb{N}$ be a number such that $n \geq 2$. Prove the zero-divisor property for the integers modulo $n$, i.e., $n$ is prime if and only if for all $k, l \in \mathbb{Z}$.
\begin{center}
 $kl \equiv 0$ mod $n$ implies $k \equiv 0$ mod $n$ or $l \equiv 0$ mod $n$.
 \end{center}
\begin{proof}
  This is what is called \textbf{Euclid's Lemma}. The proof relies on \textbf{Theorem 2.5} from slide 10.\\
 Let's suppose $n$ is prime and $kl \equiv 0$ mod $n$. We can then assume that gcd($n$, $k$)=1. In that case, according to \textbf{Theorem 2.5}, there exists such $x, y \in \mathbb{Z}$ for which $nx + ky = 1$. When we multiply the whole equation with $l$ we get $l(nx + ky) = l$ or $nxl + kyl = l$. Since $n \equiv 0$ mod $n$ and $kl \equiv 0$ mod $n$, we have that $(nxl + kyl) \equiv 0$ mod $n$, or in other words $l \equiv 0$ mod $n$, which is what was supposed to be proven.\\
 Now let's suppose the opposite, that $n$ is not a prime number. That would mean that $n$ is composite. Since $k$ and $l$ can be any numbers, we can have a case where $n=kl$ (which makes it so that $kl \equiv 0$ mod $n$). In this case we have $k \not\equiv 0$ mod $n$ and $l \not\equiv 0$ mod $n$. \\
 So we can conclude that $n$ is prime if and only if for all $k, l \in \mathbb{Z}$ $kl \equiv 0$ mod $n$ implies $k \equiv 0$ mod $n$ or $l \equiv 0$ mod $n$.
\end{proof}


\noindent\rule{12cm}{0.4pt}\\
\noindent \textbf{Exercise 2.2} Use the Chinese Remainder Theorem to find a solution (or show one does not exist) for the following sets of equations.
\begin{center}
(1) $x \equiv 3$ mod $2879$, $x \equiv 4$ mod $3541$, $x \equiv 2$ mod $5333$;\\
(2) $x \equiv 10$ mod $19271$, $x \equiv 3$ mod $7343$, $x \equiv 8$ mod $9973$.
\end {center}
\begin{proof}
Let's begin by solving the first problem:
\begin{center}
(1)\\
$x \equiv 3$ mod $2879$, $x \equiv 4$ mod $3541$, $x \equiv 2$ mod $5333$\\
$r_1 = 3$, $r_2 = 4$, $r_3 = 2$\\
$M = 2879 \cdot 3541 \cdot 5333=54367476487$\\
$M_1 = 18884153$, $M_2 = 15353707$, $M_3 = 10194539$\\
$18884153x_1 \equiv 1$ mod $2879$, $15353707x_2 \equiv 1$ mod $3541$, $10194539x_3 \equiv 1$ mod $5333$ $\implies$ $x_1 = 1574$, $x_2 = 1129$, $x_3 = 717$\\
$x = r_{1}M_{1}x_{1} + r_{2}M_{2}x_{2} + r_{3}M_{3}x_{3}$ $\implies$\\
$x = 3 \cdot 18884153 \cdot 1574 + 4 \cdot 15353707 \cdot 1129 + 2 \cdot 10194539 \cdot 717$ $\implies$\\
$x = 103851354220$
\end{center}
And for the second one:
\begin{center}
(2)\\
$x \equiv 10$ mod $19271$, $x \equiv 4$ mod $7343$, $x \equiv 8$ mod $9973$\\
$r_1 = 10$, $r_2 = 4$, $r_3 = 8$\\
Since $19271$ and $7343$ are not relatively prime (They are both divisible by 7), we can rewrite the system as: \\
$x \equiv 10$ mod $2753$, $x \equiv 10$ mod $7$, $x \equiv 4$ mod $1049$, $x \equiv 4$ mod $7$, $x \equiv 8$ mod $9973$\\
However, since $10 \not \equiv 4$ mod 7, there are no solutions to this system.
\end{center}
\end{proof}



\noindent\rule{12cm}{0.4pt}\\
\noindent \textbf{Exercise 2.3} Show the following:
Let $a, c \in \mathbb{Z}$ and let $b \in \mathbb{N}, b \geq 2$. Then the linear congruence $ax \equiv c$ mod $b$ is solvable if and only if gcd($a$,$b$) divides $c$. Moreover if $d$ = gcd($a$,$b$) and $d \mid c$ then there are precisely $d$ solutions in $\mathbb{Z}_{b}$ for $ax \equiv c$ mod $b$.
\begin{proof}
By \textbf{Euclidean Algorithm} we have that there are $y, z \in \mathbb{Z}$ such that $ay + bz = 1$, so after multiplying by $c$ we have that $ayc + bzc = c$. Since $ax \equiv c$ mod $b$ means that $ax$ equals $c$ plus some multiply of $b$, we can rewrite it as $ax = c + bk$ for some $k \in  \mathbb{Z}$. In this case, $x = yc$ and $k = -zc$. The existence of solutions $y$ and $z$ requires that there is $d$ = gcd($a$, $b$) that divides $c$, which is what needed proof. \\
Additionally, from $ax - bk = c$ it is visible that if $d \nmid c$ there are no solutions. Now all that is left is to show that there are precisely $d$ solutions in $\mathbb{Z}_b$. If we have one solution $x_0$ we can write the equation:
\begin{center}
$x = x_0 + \frac{b}{d}t$
\end{center}
If we suppose that we have two solutions they are congruent to mod $b$:
\begin{center}
$x_0 + \frac{b}{d}t_1 \equiv x_0 + \frac{b}{d}t_2$ mod $b$ $\implies$ $\frac{b}{d}t_1 \equiv \frac{b}{d}t_2$ mod $b$
\end{center}
Since $\frac{b}{d}$ divides both sides and we have that gcd($\frac{b}{d}$, $b$) = $\frac{b}{d}$ we have:
\begin{center}
	$t_1 \equiv t_2$ mod $d$
\end{center}
Which shows that there are distinct solutions to equation $x = x_0 + \frac{b}{d}t$ as many as there are residues to the modulo of $d$, or in other words, exactly $d$ solutions.
\end{proof}



\noindent\rule{12cm}{0.4pt}\\
\noindent \textbf{Exercise 2.4} Determine the complexity of computing all solutions of $ax \equiv c$ mod $b$ in dependence on the length of the input $a, c \in \mathbb{Z}$ and $b \in \mathbb{N}, b \geq 2$.
\begin{proof}
Since according to \textbf{Theorem 2.9} from slide 33, time complexity of calculating modular inverse can be computed in time $\mathcal{O}(max\{log\ a, log\ m\}^3)$, in this case complexity would be:
\begin{center}
$\mathcal{O}(max\{log\ a, log\ c, log\ b\}^3)$
\end{center}
\end{proof}



\noindent\rule{12cm}{0.4pt}\\
\noindent \textbf{Exercise 2.5} Study the problem to compute all integer solutions of \textbf{linear Diophantine equations}, i.e., equations of the form $ax + by = c$ for $a, b, c \in \mathbb{Z}$
\begin{proof}
The Diophantine equation is the polynomial equation in which the coefficients are integers and Diophantine equations whose solutions we seek in the set of integers or natural numbers. The most basic Diophantine equation is the linear case. We can write $ax + by = c$ where $a, b, c \in \mathbb{Z}$. An example of Diophantine equation problem is given in \textbf{Exercise 2.3}. Couple of theorems related to Diophantine equations follow: \\
Let $a$,$b$, and $c$ be integers with $a$ and $b$ both not zero. The linear Diophantine equation
\begin{center}
$ax + by = c$
\end{center}
has a solution if and only if $d$ = gcd($a$, $b$) divides $c$.\\
Let $a$ and $b$ be integers with $d$ = gcd($a$, $b$). The equation $ax + by = c$ has no integral solution if $d$ doesn’t divide $c$. If $d \mid c$, then there are infinitely many integral solutions. Moreover, if $x = x_0$, $y = y_0$ is a particular solution of the equation, then all solutions are given by 
\begin{center}
$x = x_0 + \frac{b}{d}n,\ y=y_0 - \frac{a}{d}n,$
\end{center}
Where $n$ is an integer.\\
If $a_1, a_2,..., a_n$ are non zero positive integers, then the equation $a_1x_1 + a_2x_2 +...+ a_nx_n = c$ has an integral solution if and only if $d$ = ($a_1, a_2,..., a_n$) divides $c$. Furthermore, when there is a solution, there are infinitely many solutions.
\end{proof}



\noindent\rule{12cm}{0.4pt}\\
\noindent \textbf{Exercise 2.6} Hiroko has $10\ 000$ Yen and has to buy eggs, chicken pieces, or turkey. An egg costs 25 Yen, a chicken piece 100 Yen and a turkey 2500 Yen. Hiroko has to buy 100 items of at least two different types and to spend all her money. What is she buying?
\begin{proof}
Since the problem states that Hiroko needs to buy 100 items, we can write:
\begin{center}
$e+c+t = 100$	
\end{center}
Where $e$ is the number of eggs, $c$ is number of chicken pieces and $t$ is number of turkeys. Because Hiroko needs to spend all her money, we also have following equation:
\begin{center}
$25e+100c+2500t=10000$	
\end{center}
Where 25, 100 and 2500 are price expressed in Yen for eggs, chicken pieces and turkey respectively.\\
This problem is one that involves \textbf{Diophantine equations} which have been shortly described as part of \textbf{Exercise 2.5}, since we have more unknowns than we have equations. What is worth noting before we start is also a condition that Hiroko needs to buy at least 2 different types of groceries, so we can't have that two different variables are 0.\\
We start by multiplying the first equation by 100 and then subtracting it from the second one, which gives us:
\begin{center}
$-75e+2400t = 0$	
\end{center}
From here we have that $e$ is equal to:
\begin{center}
$e = 32t$	
\end{center}
Which tells us that both $e$ and $t$ can't be zero, because it would lead to Hiroko buying only chicken. First thing we can do is have $t=1$, this would lead to $e=32$ and from first equation we have that $c=67$, which causes Hiroko to spend exactly 10000 Yen, meaning this is one of the solutions. Since these equations often have multiple solutions, we can try another value of $t$. For $t=2$ we have $e=64$ and $c=34$, which when multiplied by price value of every item gives us exactly 10000 Yen, meaning it is another solution. If $t=3$, this gives us $e=96$ and $c=1$, which also when we multiply with prices gives 10000 Yen, so we get it as a third solution.\\
Due to the price of turkey being 2500 Yen, Hiroko can only buy up to 4 turkeys, but buying 4 would cause her to spend all her money on them, without buying other items. So in conclusion, this set of equations has 3 solutions:
\begin{center}
$e=32,\ c=67,\ t=1$\\
$e=64,\ c=34,\ t=2$\\
$e=96,\ c=1,\ t=3$	
\end{center}
\end{proof}



\noindent\rule{12cm}{0.4pt}\\
\noindent \textbf{Exercise 2.7} Prove or disprove: Let ($G$, $\circ$) be any Abelian group, let $a, b \in G$ and $m,n \in \mathbb{N}$ be such that ord($a$) = $n$ and ord($b$) = $m$ and gcd($m,n$) = $1$. Then ($a \circ b$) has the order $mn$.
\begin{proof}
By definition of order we have that $a^n=e$ and $b^m=e$ (\textbf{Definition 1.5}). Since gcd($m$, $n$) = 1, we can also say that $m$ and $n$ are relatively prime (\textbf{Definition 2.3}). If we power $(a \circ b)$ to $mn$ we have the following equation:
\begin{center}
$(a \circ b)^{mn}=(a^{mn} \circ b^{mn})$
\end{center}
Since group is Abelian. We can further write it as:
\begin{center}
$(a^{mn} \circ b^{mn})=((a^n)^m \circ (b^m)^n)=e \circ e=e$
\end{center}
Where $e$ is a neutral element. This implies that order of $(a \circ b)$, which we will mark as $p$ divides $mn$. Next we can write:
\begin{center}
$(a \circ b)^p=(a^{p} \circ b^{p})=e$
\end{center}
We can further write that:
\begin{center}
$e=e^n=(a^{pn} \circ b^{pn})= (e \circ b^{pn}) =b^{pn}$
\end{center}
From $b^{pn}=e=b^m$ we can tell that $m$ divides $pn$. Since $n$ and $m$ are relatively prime, we have that $m$ divides $p$. Similarly if we write:
\begin{center}
$e = e^m=(a^{pm} \circ b^{pm})= (a^{pm} \circ e) =a^{pm}$
\end{center}
This implies that $n$ which is order of $a$ divides $pm$. Since $n$ and $m$ are relatively prime, we have that $n$ divides $p$. Meaning that $mn$ divides $p$. Since $p$ divides $mn$ and $mn$ divides $p$, we have that $mn=p$, so order of $(a \circ b)$ is $mn$.
\end{proof}



\noindent\rule{12cm}{0.4pt}\\
\noindent \textbf{Exercise 2.8} Prove or disprove: Let ($G$, $\circ$) be any Abelian group, let $a \in G$ and let $k, n \in \mathbb{N}$ be such that ord($a$) = $n$. Then $a^{k}$ has the order $n/gcd(n,k)$.
\begin{proof}
Let's start by marking gcd($n$, $k$) = $d$. Then since group is Abelian we have that:
\begin{center}
$(a^k)^{\frac{n}{d}} = (a^n)^{\frac{k}{d}}= e^{\frac{k}{d}} = e$
\end{center}
This tells us that ord($a^k$) $\leq \frac{n}{d}$. Next let's assume there is $m \in \mathbb{N}$ such that $(a^k)^m=a^{km}=e$, and let it be order of $a^k$, from which we have that $n \mid km$ so we also have that $\frac{n}{d} \mid \frac{km}{d}$. Since gcd($\frac{n}{d}$, $\frac{k}{d}$) = 1 it follows that $\frac{n}{d} \div m$ from where we have that $\frac{n}{d} \leq m$. Since $\frac{n}{d}$ is both less than equal and greater than equal than order of $a^k$ we can say that ord($a^k$) = $\frac{n}{d}$ which is what needed proving.
\end{proof}



\noindent\rule{12cm}{0.4pt}\\
\noindent \textbf{Exercise 2.9} Show that $n^5 - n \equiv 0$ mod 30 for every $n \in \mathbb{N}$.
\begin{proof}
If we take that $n \in \mathbb{N}$ and factorize the expression we get:
\begin{center}
$n^5 - n = n \cdot (n^2 - 1) \cdot (n^2 + 1) = n \cdot (n - 1) \cdot (n + 1) \cdot (n^2 + 1)$	
\end{center}
Since the expression has product of three consecutive integers, namely $n$, $n+1$ and $n-1$ we can tell that the expression is divisible by both 2 and 3, meaning that it is also divisible by their product which is 6.\\
In order to prove that $n^5 - n$ is also divisible by 5, which is needed to prove that it is always divisible by 30 (Since it is divisible by both 6 and 5, it is then divisible by 30), we can go about this 2 ways. First one is using \textbf{Fermat's Little Theorem}\footnote{Fermat's Little Theorem - https://mathworld.wolfram.com/FermatsLittleTheorem.html} which says that $a^p \equiv a$ mod $p$, if $p$ is a prime number. Since 5 is prime, it can be applied to $n^5 \equiv n$ mod 5. The other way is by further factorizing the expression $n^5 - n$ like so:
\begin{center}
$n^5 - n = n \cdot (n^2 - 1) \cdot (n^2 + 1) = n \cdot (n^2 - 1) \cdot (n^2 - 4 + 5) = n \cdot (n^2 - 1) \cdot (n^2 - 4) + 5 \cdot n \cdot (n^2 - 1) = (n - 2) \cdot (n -1) \cdot n \cdot (n + 1) \cdot (n + 2) + 5$
\end{center}
Which is a product of 5 consecutive integers and a constant that is divisible by 5. Since this is a product of 5 consecutive integers, one of them must be divisible by 5, one has to be divisible by 3 and one has to be divisible 2. Since $n^5 - n$ is divisible by 5 and 6, it is then also always divisible for 30, for any $n \in \mathbb{N}$, which is what needed proof.

\end{proof}
\end{document}