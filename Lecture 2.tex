\documentclass{article}

\usepackage{amsmath,amsthm,amssymb}

\newcommand{\subtitle}[1]{%
  \posttitle{%
    \par\end{center}
    \begin{center}\LARGE#1\end{center}
    \vskip0.5em}%
}

\begin{document}
\subtitle{Lecture II}
\noindent \textbf{Exercise 2.1} Let $n \in \mathbb{N}$ be a number such that $n \geq 2$. Prove the zero-divisor property for the integers modulo $n$, i.e., $n$ is prime if and only if for all $k, l \in \mathbb{Z}$.
\begin{center}
 $kl \equiv 0$ mod $n$ implies $k \equiv 0$ mod $n$ or $l \equiv 0$ mod $n$.
 \end{center}
\begin{proof}
  This is what is called \textbf{Euclid's Lemma}. The proof relies on \textbf{Theorem 2.5} from slide 10.\\
 Let's suppose $n$ is prime and $kl \equiv 0$ mod $n$. We can then assume that gcd($n$, $k$)=1. In that case, according to \textbf{Theorem 2.5}, there exists such $x, y \in \mathbb{Z}$ for which $nx + ky = 1$. When we multiply the whole equation with $l$ we get $l(nx + ky) = l$ or $nxl + kyl = l$. Since $n \equiv 0$ mod $n$ and $kl \equiv 0$ mod $n$, we have that $(nxl + kyl) \equiv 0$ mod $n$, or in other words $l \equiv 0$ mod $n$, which is what was supposed to be proven.\\
 Now let's suppose the opposite, that $n$ is not a prime number. That would mean that $n$ is composite. Since $k$ and $l$ can be any numbers, we can have a case where $n=kl$ (which makes it so that $kl \equiv 0$ mod $n$). In this case we have $k \not\equiv 0$ mod $n$ and $l \not\equiv 0$ mod $n$. \\
 So we can conclude that $n$ is prime if and only if for all $k, l \in \mathbb{Z}$ $kl \equiv 0$ mod $n$ implies $k \equiv 0$ mod $n$ or $l \equiv 0$ mod $n$.
\end{proof}


\noindent\rule{12cm}{0.4pt}\\
\noindent \textbf{Exercise 1.5} Use the Chinese Remainder Theorem to find a solution (or show one does not exist) for the following sets of equations.
\begin{center}
(1) $x \equiv 3$ mod $2879$, $x \equiv 4$ mod $3541$, $x \equiv 2$ mod $5333$;\\
(2) $x \equiv 10$ mod $19271$, $x \equiv 3$ mod $7343$, $x \equiv 8$ mod $9973$.
\end {center}
\begin{proof}
	
\end{proof}

\end{document}