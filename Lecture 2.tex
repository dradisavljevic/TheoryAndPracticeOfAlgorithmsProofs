\documentclass[a4paper]{article}

\usepackage{amsmath,amsthm,amssymb, titling, mdframed}

\title{\vspace{-2cm}Lecture II\vspace{-2cm}}
\date{}

\newmdtheoremenv{theorem}{Theorem}
\newmdtheoremenv{definition}{Definition}
\newmdtheoremenv{corollary}{Corollary}
\newmdtheoremenv{algorithm}{Algorithm}
\newmdtheoremenv{lemma}{Lemma}

\begin{document}
\maketitle
\section{Lecture Summary}
\begin{theorem}
The addition of two members $a$ and $b$ each having at most $n$ bits can be performed in time $O(n)$.
\end{theorem}


\begin{definition}
The value of an AS-number $x = (x_{n-1}, \ldots , x_{0}) \in \lbrace 0,1 \rbrace^{n}$ is defined as
\begin{center}
$V_{AS}(x) =_{df} (-1)^{x_{n-1}}(x_{n-2}2^{n-2}+ \ldots + x_{1}2 + x_{0})$.
\end{center}
\end{definition}


\begin{theorem}
The subtraction of two AS-numbers $a$ and $b$ each having at most $n$ bits can be performed in time $O(n)$.
\end{theorem}


\begin{theorem}
The usual school algorithm for multiplying two numbers $a$ and $b$ each having at most $n$ bits can be performed in time $O(n^{2})$.
\end{theorem}


\begin{theorem}
There is an algorithm which, on input any two numbers $a$ and $d$ having at most $n$ bits, precisely computes the quotient $a/d$ for $n$ bits and/or performs division with remainder in time $O(M(n))$.
\end{theorem}


\begin{theorem}
Algorithm ECL computes the gcd $d$ of numbers $a, b \in \mathbb{N}^{+}$ and numbers $x, y \in \mathbb{Z}$ such that $d = ax + by$. It uses at most $1.5\ log \ a$ many divisions of numbers less than or equal to $a$.
\end{theorem}


\begin{algorithm}
\textbf{Algorithm ECL -}
\begin{center}
Set $x_{0} = 1, x_{1} = 0, y_{0} = 0, y_{1} = 1$, and $r_{0} = a, r_{1} = b$. \\
For $i = 1, 2, \ldots$ compute successively \\
$r_{i+1} = r_{i-1} - q_{i}r_{i}$, where $q_{i} = \lfloor \frac{r_{i}-1}{r_{i}} \rfloor$, \\
$x_{i+1} = x_{i-1} - q_{i}x_{i}$, and \\
$y_{i+1} = y_{i - 1} - q_{i}y_{i}$ until $r_{i+1} = 0$. \\
Output $r_{i}, x_{i}, y_{i}$.
\end{center}
\end{algorithm}


\begin{theorem}
During the execution of Algorithm ECL we always have $|x_{\mathit{l}}| \leq b / (2d)$ and $|y_{\mathit{l}}| \leq a/(2d) for\ \mathit{l} = 0, \ldots, i$ where $i$ is the smallest number such that $r_{i+1} = 0$.
\end{theorem}


\begin{theorem}
The time complexity of Algorithm ECL is $O((\log{a})^{3})$.
\end{theorem}


\begin{theorem}
The congruence $ax \equiv 1$ mod $m$ is solvable iff gcd($a$, $m$) = $1$. Moreover if $ax \equiv 1$ mod $m$ is solvable, then the solution is uniquely determined.
\end{theorem}


\begin{theorem}
Modular inverse can be computed in time $O(max\{log{a}, log{m}\}^{3})$.
\end{theorem}


\begin{definition}
A number $n \in \mathbb{N}$ is said to be prime if $n \geq 2$ and if $n$ is only divisible by itself and by $1$. If $n$ is not a prime number, we call it composite.
\end{definition}


\begin{corollary}
For every $n \in \mathbb{N}$ we have: The structure ($\mathbb{Z}_{n}, +, \cdot$) is a finite Abelian field iff $n$ is prime.
\end{corollary}


\begin{definition}
Integers $a$ and $b$ are said to be relatively prime if gcd($a, b$) = $1$. Integers $m_{1}, \ldots, m_{r}$ are said to be pairwise relatively prime if every pair $m_{i}, m_{j}, i \neq j$, is relatively prime.
\end{definition}


\begin{theorem}
\textbf{Chinese Remaindering -}
Let $m_{1}, \ldots, m_{r}$ be pairwise relatively prime numbers, let $M = \prod^{r}_{i=1}m_{i}$, and let $a_{1}, \ldots, a_{r}$ be any integers. Then there is a unique $y \in \mathbb{Z}_{M}$ such that $ y \equiv a_{i}$ mod $m_{i}$ for $i = 1, \ldots, r$. Moreover $y$ can be computed in time polynomial in the length of the input.
\end{theorem}


\begin{definition}
A function f: $\mathbb{N} \rightarrow \mathbb{N}$ is said to be multiplicative if $f(1) = 1$ and $f(mn) = f(m)f(n)$ for all $m, n \in \mathbb{N}$ wherever gcd($m, n$) = $1$.
\end{definition}


\begin{theorem}
(1) $\varphi (mn) = \varphi (m) \varphi (n)$ if gcd($m, n$) = $1$; \\
(2) $\varphi (p^{\alpha}) = p^{\alpha - 1}(p - 1)$ if $p$ is prime and $\alpha \geq 1$; \\
(3) $\varphi(p) = p - 1$ if and only if $p$ is prime
\end{theorem}


\begin{theorem}
For all $n \in \mathbb{N}^{+}$ we have $\sum_{d|n} \varphi (d) = n$.
\end{theorem}



\section{Exercises}
\noindent \textbf{Exercise 2.1} Let $n \in \mathbb{N}$ be a number such that $n \geq 2$. Prove the zero-divisor property for the integers modulo $n$, i.e., $n$ is prime if and only if for all $k, l \in \mathbb{Z}$.
\begin{center}
$kl \equiv 0$ mod $n$ implies $k \equiv 0$ mod $n$ or $l \equiv 0$ mod $n$.
\end{center}
\begin{proof}
We start with the cancellation law for the integers mod $n$, i.e.
$$ \forall j \forall k \forall l[j \not \equiv 0\ mod\ n\ and\ jk \equiv jl\ mod\ n\ implies\ k \equiv l\ mod\ n]$$
holds if and only if $n$ is prime. \\
If $n$ is not prime we can find a counterexample to the cancellation law. Let $jk$ be a non-trivial factorization of $n$; i.e., $n = jk$ and $1 < j$, $k < n$, and let $l = 0$. Both $jk$ and $jl$ are congruent to $0$ modulo $n$, but $k \not \equiv l$ mod $n$.\\
It remains to prove that the cancellation law holds if $n$ is prime. Assume that $n$ is prime and $j \not \equiv 0$ mod $n$. In this case gcd($n, j$) = $1$, and we can find an integer $x$ such that $xj \equiv 1$ mod $n$. If $jk \equiv jl$ mod $n$ by multiplying both sides by $x$ we obtain
\begin{center}
$xjk \equiv xjl$ mod $n$\\
$k \equiv l$ mod $n$
\end{center}
completing the proof of the cancellation law.\\
To show the zero-divisor property, first consider a composite number $n$ and let $kl$ be a non-trivial factorization of $n$; i.e., $n = kl$ and $1 < k, l <n$. For such $k$ and $l$, we have $kl \equiv 0$ mod $n$ and neither $k$ nor $l$ is congruent to $0$ modulo $n$. Therefore the cancellation law does not hold.\\
Assume that $n$ is prime and let $k$ and $L$ be two numbers such that $kl \equiv 0$ mod $n$. Then $n|kl$ and since $n$ is prime, either $n|k$ or $n|l$. Therefore, either $k \equiv 0$ mod $n$ or $l \equiv 0$ mod $n$. This proves the zero-divisor property.
\end{proof}


\noindent\rule{12cm}{0.4pt}\\
\noindent \textbf{Exercise 2.2} Use the Chinese Remainder Theorem to find a solution (or show one does not exist) for the following sets of equations.
\begin{center}
(1) $x \equiv 3$ mod $2879$, $x \equiv 4$ mod $3541$, $x \equiv 2$ mod $5333$;\\
(2) $x \equiv 10$ mod $19271$, $x \equiv 3$ mod $7343$, $x \equiv 8$ mod $9973$.
\end {center}
\begin{proof}
We start with Part (1). First we compute the gcd($3541,2879$) = $1$, gcd($5333, 3541$) = $1$, and gcd($5333, 2879$) = $1$. Hence, the assumptions of \textbf{Theorem 10} are satisfied. We have $m_{1} = 2879$, $m_{2} = 3541$ and $m_{3} = 5333$. \\
Next we calculate $M = 2879 \cdot 3541 \cdot 5333 = 54367476487$. Furthermore, we directly obtain $n_{1} = 54367476487 / 2879 = 18884153$, $n_{2} = 54367476487 / 3541 = 15353707$, and $n_{3} = 54367476487 / 5333 = 10194539$. Then by using the Algorithm ECL, we compute the modular inverse $n_{i}^{-1}$ of $n_{i}$ mod $m_{i}$ for $i = 1,2,3$ and obtain \\
\begin{center}
$n_{1}^{-1} = 1547$, $n_{2}^{-1} = 1129$, $n_{3}^{-1} = 717$
\end{center}
Finally, we have to compute
\begin{center}
\begin{tabular}{c c c}
$y$ & $\equiv$ & $\sum_{i=1}^{3}{n_{i}n_{i}^{-1}a_{i}}$ \\
& $\equiv$ & $18884153 \cdot 1574 \cdot 3 + 15353707 \cdot 1129 \cdot 4 + 10194539 \cdot 717 \cdot 2$ \\
& $\equiv$ & $10024850743$ mod $54367476487$
\end{tabular}
\end{center}
where $a_{1} = 3$, $a_{2} = 4$ and $a_{3} = 2$. A quick check is in order here. We easily verify that $10024850743 \equiv 3$ mod $2879$, $10024850743 \equiv 4$ mod $3541$, and finally, we also have $10024850743 \equiv 2$ mod $5333$. \\
For Part (2), we observe that gcd($19271, 7343$) = 7. Hence the assumptions of \textbf{Theorem 10} are not satisfied. Does this mean that there is no solution of the system of simultaneous congruences given here? \\
To answer this question, first we look at two congruences; i.3.,
\begin{center}
$x \equiv a_{1}$ mod $m_{1}$, $x \equiv a_{2}$ mod $m_{2}$,
\end{center}
where gcd($m_{1}, m_{2}$) = $d > 1$. \\
Claim 1. If the system above is solvable then $a_{1} \equiv a_{2}$ mod d. \\
Let $x$ be a solution to the system. Then we conclude that $m_{1}|(x-a_{1})$ and $m_{2}|(x-a_{2})$. By the transitivity of the divisibility relation and the choice of $d$ we thus have $d|(x-a_{1})$ and $d|(x-a_{2})$; i.e., we have $x \equiv a_{1}$ mod $d$ and $x \equiv a_{2}$ mod d. Consequently, we arrive at $a_{1} \equiv a_{2}$ mod $d$, and Claim 1 is shown. \\
Claim 2. If $a_{1} \equiv a_{2}$ mod $d$ then the system is solvable. \\
Since $d$ = gcd($m_{1}$, $m_{2}$), by \textbf{Theorem 5} we know that there are integers $l, m \in \mathbb{Z}$ such that $d = l \cdot m_{1} + m \cdot m_{2}$. Furthermore $a_{1} \equiv a_{2}$ mod $d$ impl;ies $a{2} \equiv a{1}$ mod $d$ (symmetry of the congruence relation). Hence there is a $t \in \mathbb{Z}$ such that $td = a_{2} - a_{1}$. We conclude that
$$t(l \cdot m_{1} + m \cdot m_{2}) = a_{2} - a_{1} $$
$$t \cdot l \cdot m_{1} + a_{1} = -t \cdot m \cdot m_{2} + a_{2}$$
Let $x =_{df} t \cdot l \cdot m_{1} + a_{1}$. We see that $x - a_{1} = t \cdot l \cdot m_{1}$; i.e., we have $x \equiv a_{1}$ mod $m_{1}$. By construction we also know that $x = -t \cdot m \cdot m_{2} + a_{2}$, and therefore we conclude $x \equiv a_{2}$ mod $m_{2}$. Thus $x$ is a solution of the system.\\
Let us continue with an example. We consider
$$ x \equiv 5\ mod\ 6,\ x \equiv 7\ mod\ 14$$
We have gcd($14,6$) = $2$ and since $14 = 2 \cdot 6 + 2$, we conclude that $l = -2$ and $m = 1$. Furthermore, $t = 1$ and so $x = -7$. Note that $6 \cdot 14 = 84$. Thus our solution has to be taken modulo $84$, and we have $x \equiv 77$ mod $84$. We easily verify that
$$x \equiv 77 \equiv 5\ mod\ 6\ and\ x \equiv 77 \equiv 7\ mod\ 14.$$
Moreover lcm($14, 6$) = $42$ and $77 \equiv 35$ mod $42$. So, in this example we have found two solutions. There are no further solutions since the solution modulo $42$ is uniquely determined by \textbf{Theorem 10}.\\
Let us come back to the exercise problem. We saw that gcd($19271, 7343$) = $7$. Using the \textbf{Algorithm 1}, we compute $l, m \in \mathbb{Z}$ such that $7 = l \cdot 19271 + m \cdot 7343$, i.e., we have $l = 418$ and $m = -1097$. Furthermore, we have $d = 7$ and check whether or not $3 \equiv 10$ mod $7$. Since this is true, we also obtain $t = -1$. Note that $19271 \cdot 7343 = 141506953$ and lcm($19271, 7343$) = $20215279$. As above, we arrive at
$$x \equiv -8055268 \equiv 133451685\ mod\ 141506953$$
Moreover, we obtain
$$133451685 \equiv 12160011\ mod\ 20215279$$
Hence we have the following solutions $x_{i} = 12160011 + i \cdot 20215279$ mod $141506953$ for $i = 0, 2, \ldots, 6$; i.e,
\begin{center}
\begin{tabular}{c c}
$x_{1} = 12160011,$ & $x_{5} = 93021127$, \\
$x_{2} = 32375290,$ & $x_{6} = 113236406$, \\
$x_{3} = 52590569,$ & $x_{7} = 133451685$, \\
$x_{4} = 72805848,$ &  \\
\end{tabular}
\end{center}
Next, we consider the third congruence of this exercise, i.e., $x \equiv 8$ mod $9973$ and compute gcd($19271, 9973$) = $1$ as well as gcd($9973, 7343$) = $1$.\\
It remains to find the solutions of the system
\begin{center}
$x \equiv 10$ mod $19271$, $x \equiv 3$ mod $7343$, $x \equiv 8$ mod $9973$.
\end{center}
This is done as follows: We reduce this system of three simultaneous congruences to a system of two simultaneous congruences by considering $x \equiv x_{i}$ mod $141506953$ and $x \equiv 8$ mod $9973$ for every fixed $i \in \{0, \ldots, 6\}$. Since gcd($141506953, 9973$) = $1$ this system has a uniquely determined solution $y$ and we can use the general method provided in the proof of \textbf{Theorem 10} to compute it.\\
Thus, we have $M = 141506953 \cdot 9973 = 1411248842269$, and obtain $n_{i} = 9973$ as well as $n_{2} = 141506953$. The modular inverse $n_{1}^{-1}$ of $n_{1}$ modulo $m_{1} = 141506953$ is $113710691$ and the modular inverse $n_{2}^{-1}$ of $n_{2}$ modulo $m_{2} = 9973$ is $1959$. Hence we obtain (by using $x_{1}$)
\begin{center}
\begin{tabular}{c c c}
$y_{1}$ & $\equiv$ & $9973 \cdot 113710691 \cdot 12160011 + 141506953 \cdot 1959 \cdot 8$ \\
& $\equiv$ &  $295054157016$ mod $1411248842269$.
\end{tabular}
\end{center}
Analogously, we compute the remaining six solutions by using $x_{i}$ for $i = 2, \ldots, 7$ and have
\begin{center}
\begin{tabular}{c c}
$y_{2} = 899875089417$ , & $y_{5} = 1303089044351$ , \\
$y_{3} = 93447179549$ , & $y_{6} = 496661134483$ , \\
$y_{4} = 698268111950$ , & $y_{7} = 1101482066884$ .
\end{tabular}
\end{center}
\end{proof}



\noindent\rule{12cm}{0.4pt}\\
\noindent \textbf{Exercise 2.3} Show the following:
Let $a, c \in \mathbb{Z}$ and let $b \in \mathbb{N}, b \geq 2$. Then the linear congruence $ax \equiv c$ mod $b$ is solvable if and only if gcd($a$,$b$) divides $c$. Moreover if $d$ = gcd($a$,$b$) and $d \mid c$ then there are precisely $d$ solutions in $\mathbb{Z}_{b}$ for $ax \equiv c$ mod $b$.
\begin{proof}
We start with the sufficiency. Let $d = gcd(a,b)$ and assume that $d|c$. Consider $\tilde{a} = a/d$, $\tilde{b} = b/d$, $\tilde{c} = c/d$, and $\tilde{a}x \equiv \tilde{c}$ mod $\tilde{b}$. Since $gcd(\tilde{a}, \tilde{b}) = 1$, we can apply \textbf{Theorem 8} and conclude that there is number $y$ such that
$$\tilde{a}y \equiv 1\ mod\ \tilde{b}.$$
Consequently, multiplying the previous equation with $\tilde{c}$ yields $\tilde{a}y\tilde{c} \equiv mod \tilde{b}$, and thus
$$\tilde{a}x_{0} \equiv \tilde{c}\ mod\ \tilde{b},$$
where $x_{0} =_{df} y\tilde{c}$. Hence, there is $k \in \mathbb{Z}$ such that
$$k\tilde{b} = \tilde{a}x_{0} - \tilde{c}.$$
Multiplying both sides of the above equation by $d$ yields $k\tilde{b}d = \tilde{a}dx_{0} - \tilde{c}d$. Therefore, $kb = ax_{0} - c$ (recall that $\tilde{b}d = b$ and $\hat{c}d = c$). This means $ax_{0} \equiv c$ mod $b$. Consequently $x_{0}$ is also a solution of $ax \equiv c$ mod $b$.\\
The remaining ($d - 1$) solutions of $ax \equiv c$ mod $b$ are obtained as follows: We set $x_{j} =_{df} x_{0} = j\tilde{b}$ for $j = 1, \ldots, d - 1$. So $x_{0} < x_{0} + \tilde{b} < \cdots < x_{0} + (d - 1)\tilde{b}$, and thus $x_{0}, x_{0} + \tilde{b}, \ldots, x_{0} + (d - 1)\tilde{b}$ are pairwise incongruent modulo $b$.\\
Since $j\tilde{b} \equiv 0$ mod $\tilde{b}$ for all $j \in \mathbb{Z}$, we also have $\tilde{a}(x_{0}+j\tilde{b}) \equiv \tilde{c}$ mod $\tilde{b}$, and consequently there are $k_{j}, j = 1, \ldots, d-1$, such that $k_{j}\tilde{b} = \tilde{a}(x_{0}+j\tilde{b}) - \tilde{c}$.\\
Next, we multiply both sides of the latter equality by $d$ and obtain
$$ k_{j}b = a(x_{0} + j\tilde{b}) - c,$$
which implies $a(x_{0} + j \tilde{b}) \equiv c$ mod $b$. Thus $x_{0}, x_{0} + b, \ldots, x_{0} + (d - 1)\tilde{b}$ are all solutions of $ax \equiv c$ mod $b$.\\
We claim that there are no other solutions. Suppose there is a $z$ with
\begin{center}
\begin{tabular}{c c c}
$az$ & $\equiv$ & $c$ mod $b$ \\
$z$ & $\not \equiv$ & $x_{0} + j\tilde{b}$ mod $b$ for all $j = 0, \ldots, d-1$.
\end{tabular}
\end{center}
Now the congruence $az \equiv c$ mod $b$ implies $\tilde{a}z \equiv \tilde{c}$ mod $\tilde{b}$, and since $gcd(\tilde{a}, \tilde{b}) = 1$, by $\tilde{a}x_{0} \equiv \tilde{c}\ mod\ \tilde{b}$, we have $z \equiv x_{0}$ mod $\tilde{b}$. Hence, $z = x_{0} + l \tilde{b}$. Finally, since $d \tilde{b} = b$, we can conclude that $l \in \{0, \ldots, d - 1 \}$, a contradiction to previous statements. Consequently, there are precisely $d$ different solutions of $ax \equiv c$ mod $b$.\\
Necessity. Assume that $ax \equiv c$ mod $b$ is solvable. We have to show that $gcd(a, b)$ divides $c$. Let $z$ be a solution of $ax \equiv c$ mod $b$, i.e., we have $az \equiv c$ mod $b$. Thus, there must be a $k \in \mathbb{Z}$ such that $kb = az - c$. But this means $kb - az = -c$ and consequently $gcd(a,b)$ divides $c$.
\end{proof}



\noindent\rule{12cm}{0.4pt}\\
\noindent \textbf{Exercise 2.4} Determine the complexity of computing all solutions of $ax \equiv c$ mod $b$ in dependence on the length of the input $a, c \in \mathbb{Z}$ and $b \in \mathbb{N}, b \geq 2$.
\begin{proof}
Since according to \textbf{Theorem 9} from slide 33, time complexity of calculating modular inverse can be computed in time $\mathcal{O}(max\{log\ a, log\ m\}^3)$, in this case complexity would be:
\begin{center}
$\mathcal{O}(max\{log\ a, log\ c, log\ b\}^3)$
\end{center}
\end{proof}



\noindent\rule{12cm}{0.4pt}\\
\noindent \textbf{Exercise 2.5} Study the problem to compute all integer solutions of \textbf{linear Diophantine equations}, i.e., equations of the form $ax + by = c$ for $a, b, c \in \mathbb{Z}$
\begin{proof}
We apply our knowledge about linear congruences to the problem of computing all integer solutions of linear Diophantine equations; i.e., equations of the form $ax + by = c$ for $a, b, c \in \mathbb{Z}$. Note that the word Diophantine refers to Diophantus of Alexandria, who intensively studied such equations and more general ones.\\
Observe that $ax + by = c$ implies $ax \equiv c$ mod $b$ as well as $by \equiv c$ mod $a$. So we solve $ax + by = c$ by first solving $ax \equiv c$ mod $b$ or $by \equiv c$ mod $a$. Without loss of generality, we consider\\
\begin{center}
$ax \equiv c$ mod $b$.
\end{center} 
We have to compute $d = gcd(a,b)$. If $d$ does not divide $c$, we are done, since then there is no solution. If it does, there are $d$ solutions of the above equation (coming from \textbf{Exercise 2.4}). Let $d, \hat{x}, \hat{y}$ be the integers obtained by \textbf{Algorithm 1} on input $a, b$, i.e., $d = a \hat{x} + b \hat{y}$. We set $x_{0} = \frac{\hat{x}c}{d}$ mod $b$ as well as
\begin{center}
$x_{0, j} = \left( x_{0} + j \cdot \frac{b}{d} \right)$ mod $b$ , $j = 0, \ldots, d - 1$
\end{center}
We have $ax_{0,j} \equiv c$ mod $b$ for all $j = 0, \ldots, d- 1$. For each $x_{0,j}$, consider the residue class it generates, i.e., all integers $x_{0,j} + k \cdot b$, where $k \in \mathbb{Z}$. Hence,
\begin{center}
$\frac{a(x_{0,j} + k \cdot b) - c}{b} \in \mathbb{Z}$ , and therefore \\
$a(x_{0, j} + k \cdot b) - \frac{a(x_{0, j} + k \cdot b) - c}{b} \cdot b = c$
\end{center}
Consequently all solutions of $ax + by = c$ are given by
\begin{center}
$\left\lbrace \left( x_{0, j} + k \cdot b, - \frac{a(x_{0,j} + k \cdot b) - c}{b} \right) | k \in \mathbb{Z}, j = 0, \ldots, d - 1 \right\rbrace$.
\end{center}
Thus, we have characterized the solution set of $ax + by = c$.
\end{proof}



\noindent\rule{12cm}{0.4pt}\\
\noindent \textbf{Exercise 2.6} Hiroko has $10\ 000$ Yen and has to buy eggs, chicken pieces, or turkey. An egg costs 25 Yen, a chicken piece 100 Yen and a turkey 2500 Yen. Hiroko has to buy 100 items of at least two different types and to spend all her money. What is she buying?
\begin{proof}
Since the problem states that Hiroko needs to buy 100 items, we can write:
\begin{center}
$e+c+t = 100$	
\end{center}
Where $e$ is the number of eggs, $c$ is number of chicken pieces and $t$ is number of turkeys. Because Hiroko needs to spend all her money, we also have following equation:
\begin{center}
$25e+100c+2500t=10000$	
\end{center}
Where 25, 100 and 2500 are price expressed in Yen for eggs, chicken pieces and turkey respectively.\\
This problem is one that involves \textbf{Diophantine equations} which have been shortly described as part of \textbf{Exercise 2.5}, since we have more unknowns than we have equations. What is worth noting before we start is also a condition that Hiroko needs to buy at least 2 different types of groceries, so we can't have that two different variables are 0.\\
We start by multiplying the first equation by 100 and then subtracting it from the second one, which gives us:
\begin{center}
$-75e+2400t = 0$	
\end{center}
From here we have that $e$ is equal to:
\begin{center}
$e = 32t$	
\end{center}
Which tells us that both $e$ and $t$ can't be zero, because it would lead to Hiroko buying only chicken. First thing we can do is have $t=1$, this would lead to $e=32$ and from first equation we have that $c=67$, which causes Hiroko to spend exactly 10000 Yen, meaning this is one of the solutions. Since these equations often have multiple solutions, we can try another value of $t$. For $t=2$ we have $e=64$ and $c=34$, which when multiplied by price value of every item gives us exactly 10000 Yen, meaning it is another solution. If $t=3$, this gives us $e=96$ and $c=1$, which also when we multiply with prices gives 10000 Yen, so we get it as a third solution.\\
Due to the price of turkey being 2500 Yen, Hiroko can only buy up to 4 turkeys, but buying 4 would cause her to spend all her money on them, without buying other items. So in conclusion, this set of equations has 3 solutions:
\begin{center}
$e=32,\ c=67,\ t=1$\\
$e=64,\ c=34,\ t=2$\\
$e=96,\ c=1,\ t=3$	
\end{center}
\end{proof}



\noindent\rule{12cm}{0.4pt}\\
\noindent \textbf{Exercise 2.7} Prove or disprove: Let ($G$, $\circ$) be any Abelian group, let $a, b \in G$ and $m,n \in \mathbb{N}$ be such that ord($a$) = $n$ and ord($b$) = $m$ and gcd($m,n$) = $1$. Then ($a \circ b$) has the order $mn$.
\begin{proof}
By definition of order we have that $a^n=e$ and $b^m=e$ (\textbf{Definition 5}). Since gcd($m$, $n$) = 1, we can also say that $m$ and $n$ are relatively prime (\textbf{Definition 3}). If we power $(a \circ b)$ to $mn$ we have the following equation:
\begin{center}
$(a \circ b)^{mn}=(a^{mn} \circ b^{mn})$
\end{center}
Since group is Abelian. We can further write it as:
\begin{center}
$(a^{mn} \circ b^{mn})=((a^n)^m \circ (b^m)^n)=e \circ e=e$
\end{center}
Where $e$ is a neutral element. This implies that order of $(a \circ b)$, which we will mark as $p$ divides $mn$. Next we can write:
\begin{center}
$(a \circ b)^p=(a^{p} \circ b^{p})=e$
\end{center}
We can further write that:
\begin{center}
$e=e^n=(a^{pn} \circ b^{pn})= (e \circ b^{pn}) =b^{pn}$
\end{center}
From $b^{pn}=e=b^m$ we can tell that $m$ divides $pn$. Since $n$ and $m$ are relatively prime, we have that $m$ divides $p$. Similarly if we write:
\begin{center}
$e = e^m=(a^{pm} \circ b^{pm})= (a^{pm} \circ e) =a^{pm}$
\end{center}
This implies that $n$ which is order of $a$ divides $pm$. Since $n$ and $m$ are relatively prime, we have that $n$ divides $p$. Meaning that $mn$ divides $p$. Since $p$ divides $mn$ and $mn$ divides $p$, we have that $mn=p$, so order of $(a \circ b)$ is $mn$.
\end{proof}



\noindent\rule{12cm}{0.4pt}\\
\noindent \textbf{Exercise 2.8} Prove or disprove: Let ($G$, $\circ$) be any Abelian group, let $a \in G$ and let $k, n \in \mathbb{N}$ be such that ord($a$) = $n$. Then $a^{k}$ has the order $n/gcd(n,k)$.
\begin{proof}
Let's start by marking gcd($n$, $k$) = $d$. Then since group is Abelian we have that:
\begin{center}
$(a^k)^{\frac{n}{d}} = (a^n)^{\frac{k}{d}}= e^{\frac{k}{d}} = e$
\end{center}
This tells us that ord($a^k$) $\leq \frac{n}{d}$. Next let's assume there is $m \in \mathbb{N}$ such that $(a^k)^m=a^{km}=e$, and let it be order of $a^k$, from which we have that $n \mid km$ so we also have that $\frac{n}{d} \mid \frac{km}{d}$. Since gcd($\frac{n}{d}$, $\frac{k}{d}$) = 1 it follows that $\frac{n}{d} \div m$ from where we have that $\frac{n}{d} \leq m$. Since $\frac{n}{d}$ is both less than equal and greater than equal than order of $a^k$ we can say that ord($a^k$) = $\frac{n}{d}$ which is what needed proving.
\end{proof}



\noindent\rule{12cm}{0.4pt}\\
\noindent \textbf{Exercise 2.9} Show that $n^5 - n \equiv 0$ mod 30 for every $n \in \mathbb{N}$.
\begin{proof}
We observe that
$$ n^{5} - n = (n - 1)n(n+1)(n^{2} +1).$$
Note that the first three factors constitute three successive numbers. Therefore, the product $(n-1)n(n+1)$ is always divisible by $2$ and by $3$. Furthermore, in this product one factor is either also divisible by $5$ or we have $n \equiv \pm 2$ mod $5$. In the first case, we are done.\\
In the second case we conclude that $n^{2} \equiv 4 \equiv -1$ mod $5$. Consequently since we also know that $1 \equiv 1$ mod $5$, we see by \textbf{Theorem 5} that $n^{2} + 1 \equiv 0$ mod $5$. So, the product is again divisible by $5$. Finally $2, 3,$ and $5$ are prime numbers and so their least common multiple is $30$. Hence, the product is always divisible by $30$.
\end{proof}
\end{document}