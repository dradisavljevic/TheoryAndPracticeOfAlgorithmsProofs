\documentclass{article}

\usepackage{amsmath,amsthm,amssymb}

\newcommand{\subtitle}[1]{%
  \posttitle{%
    \par\end{center}
    \begin{center}\LARGE#1\end{center}
    \vskip0.5em}%
}

\begin{document}
\subtitle{Lecture II}
\noindent \textbf{Exercise 2.1} Let $n \in \mathbb{N}$ be a number such that $n \geq 2$. Prove the zero-divisor property for the integers modulo $n$, i.e., $n$ is prime if and only if for all $k, l \in \mathbb{Z}$.
\begin{center}
 $kl \equiv 0$ mod $n$ implies $k \equiv 0$ mod $n$ or $l \equiv 0$ mod $n$.
 \end{center}
\begin{proof}
  This is what is called \textbf{Euclid's Lemma}. The proof relies on \textbf{Theorem 2.5} from slide 10.\\
 Let's suppose $n$ is prime and $kl \equiv 0$ mod $n$. We can then assume that gcd($n$, $k$)=1. In that case, according to \textbf{Theorem 2.5}, there exists such $x, y \in \mathbb{Z}$ for which $nx + ky = 1$. When we multiply the whole equation with $l$ we get $l(nx + ky) = l$ or $nxl + kyl = l$. Since $n \equiv 0$ mod $n$ and $kl \equiv 0$ mod $n$, we have that $(nxl + kyl) \equiv 0$ mod $n$, or in other words $l \equiv 0$ mod $n$, which is what was supposed to be proven.\\
 Now let's suppose the opposite, that $n$ is not a prime number. That would mean that $n$ is composite. Since $k$ and $l$ can be any numbers, we can have a case where $n=kl$ (which makes it so that $kl \equiv 0$ mod $n$). In this case we have $k \not\equiv 0$ mod $n$ and $l \not\equiv 0$ mod $n$. \\
 So we can conclude that $n$ is prime if and only if for all $k, l \in \mathbb{Z}$ $kl \equiv 0$ mod $n$ implies $k \equiv 0$ mod $n$ or $l \equiv 0$ mod $n$.
\end{proof}


\noindent\rule{12cm}{0.4pt}\\
\noindent \textbf{Exercise 2.2} Use the Chinese Remainder Theorem to find a solution (or show one does not exist) for the following sets of equations.
\begin{center}
(1) $x \equiv 3$ mod $2879$, $x \equiv 4$ mod $3541$, $x \equiv 2$ mod $5333$;\\
(2) $x \equiv 10$ mod $19271$, $x \equiv 3$ mod $7343$, $x \equiv 8$ mod $9973$.
\end {center}
\begin{proof}
Let's begin by solving the first problem:
\begin{center}
(1)\\
$x \equiv 3$ mod $2879$, $x \equiv 4$ mod $3541$, $x \equiv 2$ mod $5333$\\
$r_1 = 3$, $r_2 = 4$, $r_3 = 2$\\
$M = 2879 \cdot 3541 \cdot 5333=54367476487$\\
$M_1 = 18884153$, $M_2 = 15353707$, $M_3 = 10194539$\\
$18884153x_1 \equiv 1$ mod $2879$, $15353707x_2 \equiv 1$ mod $3541$, $10194539x_3 \equiv 1$ mod $5333$ $\implies$ $x_1 = 1574$, $x_2 = 1129$, $x_3 = 717$\\
$x = r_{1}M_{1}x_{1} + r_{2}M_{2}x_{2} + r_{3}M_{3}x_{3}$ $\implies$\\
$x = 3 \cdot 18884153 \cdot 1574 + 4 \cdot 15353707 \cdot 1129 + 2 \cdot 10194539 \cdot 717$ $\implies$\\
$x = 103851354220$
\end{center}
And for the second one:
\begin{center}
(2)\\
$x \equiv 10$ mod $19271$, $x \equiv 4$ mod $7343$, $x \equiv 8$ mod $9973$\\
$r_1 = 10$, $r_2 = 4$, $r_3 = 8$\\
Since $19271$ and $7343$ are not relatively prime (They are both divisible by 7), we can rewrite the system as: \\
$x \equiv 10$ mod $2753$, $x \equiv 10$ mod $7$, $x \equiv 4$ mod $1049$, $x \equiv 4$ mod $7$, $x \equiv 8$ mod $9973$\\
However, since $10 \not \equiv 4$ mod 7, there are no solutions to this system.
\end{center}
\end{proof}



\noindent\rule{12cm}{0.4pt}\\
\noindent \textbf{Exercise 2.3} Show the following:
Let $a, c \in \mathbb{Z}$ and let $b \in \mathbb{N}, b \geq 2$. Then the linear congruence $ax \equiv c$ mod $b$ is solvable if and only if gcd($a$,$b$) divides $c$. Moreover if $d$ = gcd($a$,$b$) and $d \mid c$ then there are precisely $d$ solutions in $\mathbb{Z}_{b}$ for $ax \equiv c$ mod $b$.
\begin{proof}
By \textbf{Euclidean Algorithm} we have that there are $y, z \in \mathbb{Z}$ such that $ay + bz = 1$, so after multiplying by $c$ we have that $ayc + bzc = c$. Since $ax \equiv c$ mod $b$ means that $ax$ equals $c$ plus some multiply of $b$, we can rewrite it as $ax = c + bk$ for some $k \in  \mathbb{Z}$. In this case, $x = yc$ and $k = -zc$. The existence of solutions $y$ and $z$ requires that there is $d$ = gcd($a$, $b$) that divides $c$, which is what needed proof. \\
Additionally, from $ax - bk = c$ it is visible that if $d \nmid c$ there are no solutions. Now all that is left is to show that there are precisely $d$ solutions in $\mathbb{Z}_b$. If we have one solution $x_0$ we can write the equation:
\begin{center}
$x = x_0 + \frac{b}{d}t$
\end{center}
If we suppose that we have two solutions they are congruent to mod $b$:
\begin{center}
$x_0 + \frac{b}{d}t_1 \equiv x_0 + \frac{b}{d}t_2$ mod $b$ $\implies$ $\frac{b}{d}t_1 \equiv \frac{b}{d}t_2$ mod $b$
\end{center}
Since $\frac{b}{d}$ divides both sides and we have that gcd($\frac{b}{d}$, $b$) = $\frac{b}{d}$ we have:
\begin{center}
	$t_1 \equiv t_2$ mod $d$
\end{center}
Which shows that there are distinct solutions to equation $x = x_0 + \frac{b}{d}t$ as many as there are residues to the modulo of $d$, or in other words, exactly $d$ solutions.
\end{proof}



\noindent\rule{12cm}{0.4pt}\\
\noindent \textbf{Exercise 2.4} Determine the complexity of computing all solutions of $ax \equiv c$ mod $b$ in dependence on the length of the input $a, c \in \mathbb{Z}$ and $b \in \mathbb{N}, b \geq 2$.
\begin{proof}
Sadly at this point I don't know how to prove this.
\end{proof}



\noindent\rule{12cm}{0.4pt}\\
\noindent \textbf{Exercise 2.5} Study the problem to compute all integer solutions of \textbf{linear Diophantine equations}, i.e., equations of the form $ax + by = c$ for $a, b, c \in \mathbb{Z}$
\begin{proof}
The Diophantine equation is the polynomial equation in which the coefficients are integers and Diophantine equations whose solutions we seek in the set of integers or natural numbers. The most basic Diophantine equation is the linear case. We can write $ax + by = c$ where $a, b, c \in \mathbb{Z}$. An example of Diophantine equation problem is given in \textbf{Exercise 2.3}. Couple of theorems related to Diophantine equations follow: \\
Let $a$,$b$, and $c$ be integers with $a$ and $b$ both not zero. The linear Diophantine equation
\begin{center}
$ax + by = c$
\end{center}
has a solution if and only if $d$ = gcd($a$, $b$) divides $c$.\\
Let $a$ and $b$ be integers with $d$ = gcd($a$, $b$). The equation $ax + by = c$ has no integral solution if $d$ doesn’t divide $c$. If $d \mid c$, then there are infinitely many integral solutions. Moreover, if $x = x_0$, $y = y_0$ is a particular solution of the equation, then all solutions are given by 
\begin{center}
$x = x_0 + \frac{b}{d}n,\ y=y_0 - \frac{a}{d}n,$
\end{center}
Where $n$ is an integer.\\
If $a_1, a_2,..., a_n$ are non zero positive integers, then the equation $a_1x_1 + a_2x_2 +...+ a_nx_n = c$ has an integral solution if and only if $d$ = ($a_1, a_2,..., a_n$) divides $c$. Furthermore, when there is a solution, there are infinitely many solutions.
\end{proof}



\noindent\rule{12cm}{0.4pt}\\
\noindent \textbf{Exercise 2.6} Hiroko has $10\ 000$ Yen and has to buy eggs, chicken pieces, or turkey. An egg costs 25 Yen, a chicken piece 100 Yen and a turkey 2500 Yen. Hiroko has to buy 100 items of at least two different types and to spend all her money. What is she buying?
\begin{proof}
This problem is a bit ambiguously defined. So for the time being it will remain unsolved.
\end{proof}

\end{document}