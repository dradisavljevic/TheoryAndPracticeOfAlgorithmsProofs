\documentclass[a4paper]{article}

\usepackage{amsmath,amsthm,amssymb, titling, mdframed}

\title{\vspace{-2cm}Lecture VII\vspace{-2cm}}
\date{}

\newmdtheoremenv{theorem}{Theorem}
\newmdtheoremenv{definition}{Definition}
\newmdtheoremenv{corollary}{Corollary}
\newmdtheoremenv{algorithm}{Algorithm}
\newmdtheoremenv{lemma}{Lemma}

\begin{document}
\maketitle
\section{Lecture Summary}
\begin{theorem}
\textbf{Multinomial Theorem -} 
$$(x_{1}+x_{2} + \cdots + x_{k})^{n} = \sum_{n_{1} + \cdots + n_{k} = n,  n_{i} \geq 0} \binom{n}{n_{1}, n_{2}, \ldots, n_{k}} x_{1}^{n_{1}}x_{2}^{n_{2}} \cdots x_{k}^{n_{k}}$$
\end{theorem}


\begin{theorem}
Let $p$ be a prime and let $f \in \mathbb{Z}_{p}[x]$. Then we have\\
(a) $f(x)^{p} \equiv f(x^{p})$ mod $p$, and\\
(b) $f(x)^{p^{j}} \equiv f(x^{p^{j}})$ mod $p$ for all $j \geq 0$.
\end{theorem}


\begin{theorem}
Let $p$ and $r$ be primes such that $p \neq r$. Then, in $\mathbb{Z}_{p}[x]$ we have
$$\frac{x^{r} - 1}{x - 1} = x^{r - 1} + x^{r - 2} + \cdots + x + 1 = h_{1}(x) \cdots h_{s}(x),$$
where for all $i = 1, \ldots, s$, all polynomials $h_{i} \in \mathbb{Z}_{p}[x]$ are irreducible and deg($h_{i}$) = $ord_{r}(p)$.
\end{theorem}


\begin{lemma}
If ($G, \circ$) is a cyclic group then every subgroup $U$ of $G$ is also cyclic.
\end{lemma}


\begin{lemma}
Let $G$ be a cyclic group and let $g_{1}, \ldots, g_{m}$ be elements from $G$, and let $e$ be the neutral element of $G$. Let $G^{'}$ be the subgroup generated by $g_{1}, \ldots, g_{m}$. Then we have the following: \\
If there is a $t \geq 1$ such that $g_{i}^{t} = e$ for all $i = 1, \ldots, m$ then $|G^{'}| \leq t$.
\end{lemma}


\begin{theorem}
Assume the following conditions to be satisfied: \\
($\alpha$) $n \geq 2$ is an odd natural number, \\
($\beta$) $r < n$ is a prime number such that $r \nmid n$, \\
($\gamma$) $q$ is a prime number such that $q|(r - 1)$, $q \geq 4 \sqrt{r}\log{n}$ and \\
($\delta$) $n^{(r - 1) / q} \not \equiv 1$ mod $n$. \\
Furthermore, let $L = \lfloor 2 \sqrt{r} \log{n} \rfloor$ and assume that \\
($\varepsilon$) gcd($a, n$) = $1$ for all $a, 1 \leq a \leq L$, \\
($\zeta$) $(X + a)^{n} \equiv (X^{n} + a)$ mod $(n, X^{r} - 1)$ for all $a, 1 \leq a \leq L$. \\
Then $n = p^{i}$ for a prime number $p$ and an $i \in \mathbb{N}^{+}$.
\end{theorem}


\begin{lemma}
The number $n$ possesses a prime factor $p$ such that $q|ord_{r}(p)$.
\end{lemma}


\begin{lemma}
$(X + a)^{n} \equiv (X^{n} + a)$ mod $(p, X^{r} - 1)$ for all $a$ with $1 \leq a \leq L$.
\end{lemma}


\begin{lemma}
$(X + a)^{n^{i}} \equiv \left( X^{n^{i}} + a \right)$ mod $(p, X^{r} - 1)$ for all $i \in \mathbb{N}$ and all $a$ such that $1 \leq a \leq L$.
\end{lemma}


\begin{lemma}
$(X + a)^{n^{i}p^{j}} \equiv \left( X^{n^{i}p^{j} }+ a \right)$ mod $(p, X^{r} - 1)$ for all $i, j \in \mathbb{N}$, and all $a$ such that $1 \leq a \leq L$.
\end{lemma}


\begin{theorem}
Algorithm AKS returns PRIME if and only if $n$ is prime.
\end{theorem}



\section{Exercises}
\noindent \textbf{Exercise 7}  Provide a formal proof to \textbf{Multinomial Theorem}.
\begin{proof}
\textbf{Multinomial Theorem} has the form of:\
$$(x_1 + x_2 + \cdots + x_k)^n = \sum_{n_1+\cdots +n_k = n\ n_i \geq 0}^{} \binom{n}{n_1, n_2, ..., n_k}x_{1}^{n_{1}}x_{2}^{n_{2}}\cdots x_{k}^{n_{k}}$$
We will prove this by induction. For $k=1$ we have:
$$(x_1)^n = \sum_{n_1 = n} \frac{n!}{n_1!} {x_1}^{n_1} =  \frac{n!}{n!} {x_1}^{n} = {x_1}^{n}$$
So our basis for induction holds. Next we move to the induction hypothesis where we have $k=m$:
$$(x_1 + x_2 + \cdots + x_m)^n = \sum_{n_1 + n_2 + \cdots \mathop + n_m = n} \binom{n}{n_1, n_2, ..., n_m} {x_1}^{n_1} {x_2}^{n_2} \cdots {x_m}^{n_m}$$
From which we will show that $k=m+1$ is true:
$$(x_1 + x_2 + \cdots + x_{m+1})^n = ((x_1 + x_2 + \cdots + x_m) + x_{m+1})^n$$
\begin{samepage}
Which, if we apply \textbf{Binomial Theorem}\footnote{Binomial Theorem - https://www.britannica.com/science/binomial-theorem} we can write as:
$$\sum_{j = 0}^{n} \binom{n}{j} {x_{m+ 1} }^j (x_1 + x_2 + \cdots + x_m)^{n - j}$$
\end{samepage}
Then from our induction hypothesis we get:
$$\sum_{j = 0}^{n} \binom{n}{j} {x_{m + 1} }^j \sum_{n_1 + n_2 + \cdots + n_m = n - j} \binom{n - j}{n_1, n_2, ..., n_m} {x_1}^{n_1} {x_2}^{n_2} \cdots {x_m}^{n_m}$$
And since summation is linear we get the following as the sum of sums:
$$\sum_{j = 0}^{n} \left(\sum_{n_1 + n_2 + \cdots + n_m = n - j} \binom{n}{j} \binom{n - j}{n_1, n_2, ..., n_m} {x_1}^{n_1} {x_2}^{n_2} \cdots {x_m}^{n_m} {x_{m + 1} }^j \right)$$
And after renaming the $j$ variable and collapsing the double sums we get:
$$\sum_{n_1 + n_2 + \cdots + n_m + n_{m + 1} = n} \binom{n}{n_{m + 1} } \binom{n - n_{m + 1}}{n_1, n_2, ..., n_m} {x_1}^{n_1} {x_2}^{n_2} \cdots {x_m}^{n_m} {x_{m + 1} }^{n_{m + 1} }$$
Now from the definition of \textbf{Binomial Coefficient}\footnote{Binomial Coefficient - https://mathworld.wolfram.com/BinomialCoefficient.html} we have:
$$\binom{n}{n_{m + 1} } \binom{n - n_{m + 1} } {n_1, n_2, ..., n_m} = \frac{n!}{n_{m + 1}! (n - n_{m + 1} )!} \frac{(n - n_{m + 1} )!} {n_1! \, n_2! \, \cdots n_m!}$$
$$= \frac{n!}{n_1! \, n_2! \, \cdots n_m! \, n_{m + 1}!} = \binom{n}{n_1, n_2, ..., n_m, n_{m + 1}}$$
Putting it all together we get the definition of multinomial theorem for any $k$, which we needed to prove.
\end{proof}
\end{document}