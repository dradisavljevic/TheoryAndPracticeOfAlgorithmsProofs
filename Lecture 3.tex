\documentclass{article}

\usepackage{amsmath,amsthm,amssymb}

\newcommand{\subtitle}[1]{%
  \posttitle{%
    \par\end{center}
    \begin{center}\LARGE#1\end{center}
    \vskip0.5em}%
}

\begin{document}
\subtitle{Lecture III}
\noindent \textbf{Exercise 3.1} Let $\mathbb{F}$ be a field. Then $\mathbb{Z}[x]$ is a commutative ring with 1. Furthermore, it is not difficult to see that division with remainder generalizes to polynomials as follows: Let $p$, $q \in \mathbb{F}[x]$ such that $q \neq0$. Then there are uniquely determined polynomials $u$, $r \in \mathbb{F}[x]$ such that
\begin{center}
$p = uq + r$, where deg $r<$ deg $q$.	
\end{center}
Moreover, let $p \in \mathbb{F}[x]$, then, for any $a \in F$, the substitution of $x$, by $a$ defines an element of $F$ which we denote by $p(a)$. The element $p(a)$ is obtained by performing the field operations once the substitution has been carried out, e.g., let $p \in \mathbb{Z}_{7}[x]$, say $p(x) = 6x^{2} + 5x + 1$. Then $p(1) = 6 + 5 + 1 = 5$.
\begin{proof}
	
\end{proof}

\end{document}