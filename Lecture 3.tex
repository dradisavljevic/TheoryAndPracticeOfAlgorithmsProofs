\documentclass[a4paper]{article}

\usepackage{amsmath,amsthm,amssymb, polynom, titling, mdframed}

\title{\vspace{-2cm}Lecture III\vspace{-2cm}}
\date{}

\newmdtheoremenv{theorem}{Theorem}
\newmdtheoremenv{definition}{Definition}
\newmdtheoremenv{corollary}{Corollary}
\newmdtheoremenv{algorithm}{Algorithm}

\begin{document}
\maketitle
\section{Lecture Summary}
\begin{theorem}
If ($G, \circ$) is a finite group, then every element of $G$ has finite order.
\end{theorem}


\begin{theorem}
\textbf{Euler's Theorem -} Let $n \in \mathbb{N}, n \geq 2$; then $a^{\varphi(n)} \equiv 1$ mod $n$ for all $a \in \mathbb{Z}^{*}_{n}$.
\end{theorem}


\begin{theorem}
\textbf{Fermat's Little Theorem -} Let $p$ be a prime. Then $a^{p-1} \equiv 1$ mod $p$ for all $a \in \mathbb{Z}^{*}_{p}$.
\end{theorem}


\begin{theorem}
Let $\mathbb{F}_{q}$ be any finite Abelian field such that $|\mathbb{F}_{q}| = q$. Then it its multiplicative group $\mathbb{F}_{q}^{*}$ we have $a^{q - 1} = 1$ for all $a \in \mathbb{F}^{*}_{q}$. Furthermore, the order of any element $a \in \mathbb{F}^{*}_{q}$ divides $q - 1$.
\end{theorem}


\begin{definition}
Let $\mathbb{F}$ be a field, let $p \in \mathbb{F}[x]$ be such that $p \neq 0$. If $a \in F$ is such that $p(a) = 0$ then we call $a$ a zero (or root) of $p$ in $\mathbb{F}$.
\end{definition}


\begin{theorem}
Let $\mathbb{F}$ be any field, and let $p \in \mathbb{F}[x]$ be a polynomial such that $p \neq 0$. Then the polynomial $p$ possesses at most $d$ zeros, where $d = deg\ p$.
\end{theorem}


\begin{corollary}
Let $\mathbb{F}$ be any field, let $p \in \mathbb{F}[x]$ be a polynomial such that $p \neq 0$, and let $a \in \mathbb{F}$ be such that $p(a) = 0$. Then the polynomial $(x - a)$ divides $p$.
\end{corollary}

\begin{definition}
Let $\mathbb{F}$ be any field and let $p \in \mathbb{F}[x]$ be any polynomial. We call $p$ irreducible if $p$ is not divisible by any polynomial of lower degree than deg $p$ except for constants.
\end{definition}


\begin{definition}
Let $p, q \in \mathbb{F}[x]$, then we call $d \in \mathbb{F}[x]$ a greatest common divisor of $p$ and $q$ if $d$ divides both $p$ and $q$ and if $d$ has the greatest degree among the polynomials having this property. We then write $d = gcd(p, q)$.
\end{definition}


\begin{theorem}
Let $\mathbb{F}_{q}$ be any finite Abelian field such that $|\mathbb{F}_{q}| = q$. Then $\mathbb{F}_{q}^{*}$ possesses a generator. If $g$ is a generator of $\mathbb{F}_{q}^{*}$ then $g^{l}$ is also a generator iff $gcd(l, q - 1) = 1)$. In particular there are $\varphi (q - 1)$ many different generators of $\mathbb{F}_{q}^{*}$.
\end{theorem}


\begin{corollary}
If $p$ is prime then $\mathbb{Z}_{p}^{*}$ is a cyclic group of order $p - 1$. If $g$ is a generator of $\mathbb{Z}_{p}^{*}$ then $g^{l}$ is also a generator iff $gcd(l, p - 1) = 1$. In particular there are $\varphi (p - 1)$ many different generators of $\mathbb{Z}_{p}^{*}$.
\end{corollary}

\begin{theorem}
There is a sequence of primes $q$ such that the probability that a random $g \in \mathbb{F}_{q}^{*}$ is a generator approaches zero.
\end{theorem}



\section{Exercises}
\noindent \textbf{Exercise 3.1} Let $\mathbb{F}$ be a field. Then $\mathbb{F}[x]$ is a commutative ring with 1. Furthermore, it is not difficult to see that division with remainder generalizes to polynomials as follows: Let $p$, $q \in \mathbb{F}[x]$ such that $q \neq0$. Then there are uniquely determined polynomials $u$, $r \in \mathbb{F}[x]$ such that
\begin{center}
$p = uq + r$, where deg $r<$ deg $q$.	
\end{center}
Moreover, let $p \in \mathbb{F}[x]$, then, for any $a \in F$, the substitution of $x$, by $a$ defines an element of $F$ which we denote by $p(a)$. The element $p(a)$ is obtained by performing the field operations once the substitution has been carried out, e.g., let $p \in \mathbb{Z}_{7}[x]$, say $p(x) = 6x^{2} + 5x + 1$. Then $p(1) = 6 + 5 + 1 = 5$.
\begin{proof}
Since every field satisfies associativity and commutativity of additive and multiplicative operations, possesses additive and multiplicative identity and inverse elements and satisfies distributivity of the multiplicative operation over addition we can say that every Field is a ring, while not all rings are fields. So the claim that if $\mathbb{F}$ is a field then $\mathbb{F}[x]$ is a ring is true.\\
Second part is what is known as \textbf{Nonmonic Polynomial Division Algorithm}. We are gonna prove now that polynomials $u$ and $r$ are in fact unique where deg $r$ $<$ deg $q$.\\
Let's say we have two polynomials that satisfy the equation:
\begin{center}
$r = p - uq$\\
$r' = p - u'q$	
\end{center}
And for them we have that $0 \leq $ deg $r\ <$ deg $q$ and $0 \leq $ deg $r'\ <$ deg $q$, since degrees can't be negative. We can subtract the equations and get:
\begin{center}
$r' - r	= q(u - u')$
\end{center}
From $r' - r	= q(u - u')$ we have that deg $(r' - r)$ = deg $q$ + deg $(u - u')$, which would imply that degree of either $r$ or $r'$ is greater than degree of $q$, which contradicts that deg $r$ $<$ deg $q$ or deg $r'$ $<$ deg $q$. This means that $u=u'$ and $r'=r$ which means that both $r$ and $u$ are unique polynomials.

\end{proof}



\noindent\rule{12cm}{0.4pt}\\
\noindent \textbf{Exercise 3.2} Let $p = 19$ and consider the finite group $\mathbb{Z}^{*}_{p}$. Determine the order of every element in $\mathbb{Z}^{*}_{p}$, and list all the generators of $\mathbb{Z}^{*}_{p}$.
\begin{proof}
Since $p=19$ is a prime number, we know that order of the group is ($p$ - 1) = 18 according to \textbf{Corollary 2}. Also, since $\mathbb{Z}^{*}_{p}$ is a finite group, every member also has finite order, according to \textbf{Theorem 1}. According to \textbf{Lagrange's Theorem (Theorem 3 from Lecture 1)}, order of every subgroup in $\mathbb{Z}^{*}_{p}$ divides the order of $\mathbb{Z}^{*}_{p}$. Since the numbers that divide 18 are: 1, 2, 3, 6, 9 and 18, we will check for them when we test for order.\\
Order is smallest number $n$ such that $a^n=1$, where 1 is the identity element for multiplicative group. We will now test for order of every element in $\mathbb{Z}^{*}_{p}$:\\
\begin{center}
$1^1 = 1 \implies ord(1) = 1$\\
$2^1 = 2, 2^2 = 4, 2^3 = 8, 2^6 = 7, 2^9 = 18, 2^{18} = 1$ mod 19 $\implies ord(2) = 18$\\
$3^1 = 3, 3^2 = 9, 3^3 = 8, 3^6 = 7, 3^9 = 18, 3^{18} = 1$ mod 19 $\implies ord(3) = 18$\\
$4^1 = 4, 4^2 = 16, 4^3 = 7, 4^6 = 11, 4^9 = 1$ mod 19 $\implies ord(4) = 9$\\
$5^1 = 5, 5^2 = 6, 5^3 = 11, 5^6 = 7, 5^9 = 1$ mod 19 $\implies ord(5) = 9$\\
$6^1 = 6, 6^2 = 17, 6^3 = 7, 6^6 = 11, 6^9 = 1$ mod 19 $\implies ord(6) = 9$\\
$7^1 = 7, 7^2 = 11, 7^3 = 1$ mod 19 $\implies ord(7) = 3$\\
$8^1 = 8, 8^2 = 7, 8^3 = 18, 8^6 = 1$ mod 19 $\implies ord(8) = 6$\\
$9^1 = 9, 9^2 = 5, 9^3 = 7, 9^6 = 11, 9^9 = 1$ mod 19 $\implies ord(9) = 9$\\
$10^1 = 10, 10^2 = 5, 10^3 = 12, 10^6 = 11, 10^9 = 18, 10^{18} = 1$ mod 19 $\implies ord(10) = 18$\\
$11^1 = 11, 11^2 = 7, 11^3 = 1$ mod 19 $\implies ord(11) = 3$\\
$12^1 = 12, 12^2 = 11, 12^3 = 18, 12^6 = 1$ mod 19 $\implies ord(12) = 6$\\
$13^1 = 13, 13^2 = 17, 13^3 = 12, 13^6 = 11, 13^9 = 18, 13^{18} = 1$ mod 19 $\implies ord(13) = 18$\\
$14^1 = 14, 14^2 = 6, 14^3 = 8, 14^6 = 7, 14^9 = 18, 14^{18} = 1$ mod 19 $\implies ord(14) = 18$\\
$15^1 = 15, 15^2 = 16, 15^3 = 12, 15^6 = 11, 15^9 = 18, 15^{18} = 1$ mod 19 $\implies ord(15) = 18$\\
$16^1 = 16, 16^2 = 9, 16^3 = 11, 16^6 = 7, 16^9 = 1$ mod 19 $\implies ord(16) = 9$\\
$17^1 = 17, 17^2 = 4, 17^3 = 11, 17^6 = 7, 17^9 = 1$ mod 19 $\implies ord(17) = 9$\\
$18^1 = 18, 18^2 = 1$ mod 19 $\implies ord(18) = 2$\\
\end{center}
So in total, there is one element with order 1 (1), one element with order 2 (18), 2 elements with order 3 (7, 11), 2 elements with order 6 (8, 12), 6 elements with order 9 (4, 5, 6, 9, 16 and 17) and 6 elements with order 18 (2, 3, 10, 13, 14, 15).
According to \textbf{Theorem 6}, there are $\varphi(p - 1)$ many different generators. Since up to 18 there are 6 numbers that are relatively prime to 18 (1, 5, 7, 11, 13, 17), we know that there are 6 generators of $\mathbb{Z}^{*}_{p}$. We first test for 2:
\begin{center}
[2, 4, 8, 16, 13, 7, 14, 9, 18, 17, 15, 11, 3, 6, 12, 5, 10, 1]
\end{center}
 Since all the elements of $\mathbb{Z}^{*}_{p}$ are present, we can tell that 2 is a generator. Since we know that 2 is a generator, according to \textbf{Theorem 6}, we can get other generators by raising 2 to the relatively prime numbers of 18:  $2^{5} \equiv 13$ mod 19, $2^{7} \equiv 14$ mod 19, $2^{11} \equiv 15$ mod 19, $2^{13} \equiv 3$ mod 19 and $2^{17} \equiv 10$ mod 19. So the list of generators would be: 2, 3, 10, 13, 14, 15. Which are also all the numbers with order of 18.
\end{proof}



\noindent\rule{12cm}{0.4pt}\\
\noindent \textbf{Exercise 3.3} Consider the group $\mathbb{Z}^{*}_{31}$. How many elements of this group have order 1, 7, 11, and 30, respectively?
\begin{proof}
Since 31 is a prime number, we know that order of the group is 30 according to \textbf{Corollary 2}. Also, according to \textbf{Lagrange's Theorem} order of every element in $\mathbb{Z}^{*}_{31}$ divides the order of $\mathbb{Z}^{*}_{31}$. Since order of the group is 30, elements can have order 1, 2, 3, 5, 6, 10, 15 and 30. Since 7 and 11 don't divide 30, we can immediately tell that number of elements in this group with $ord(7)$ and $ord(11)$ is equal to 0. We can also tell that the only element when raised to the power of 1 gives identity element, which is also 1, in this group is 1. So there is only 1 element in the group with order 1. All that is left is to check for order 30.\\
According to \textbf{Theorem 6}, there are $\varphi(p - 1)$ many different generators. There are 8 different numbers that are relatively prime to 31 (as a result of the $\varphi$ function - 3, 11, 12, 13, 17, 21, 22, 24, and all of them are generators, meaning they have order of 30 ($\varphi(30)=8$). (This can be tested by establishing 3 is a generator and then raising 3 to the power of those relatively prime numbers)\\
So there are 1, 0, 0 and 8 elements with order of 1, 7, 11 and 30 respectively.
\end{proof}



\noindent\rule{12cm}{0.4pt}\\
\noindent \textbf{Exercise 3.4} Let $p(x) = x^3 - 9x^2 + 23x - 15$ and let $q(x) = x^3 - 14x^2 + 59x - 70$ be polynomials from $\mathbb{Q}[x]$. Compute gcd($p$, $q$) in monic form and determine polynomials $u$, $v \in \mathbb{Q}[x]$ such that gcd($p$, $q$) = $pu = qv$.
\begin{proof}
In order to solve this problem we can apply \textbf{Euclidean algorithm} for polynomials. First we start by dividing $q(x)$ by $p(x)$:
\begin{center}
$\polylongdiv{x^3 - 14x^2 + 59x - 70}{x^3 - 9x^2 + 23x - 15}$\\
$\implies q(x) = 1 \cdot p(x) - (5x^2 - 36x + 55)$
\end{center}
Where $(5x^2 - 36x + 55)$ is the remainder. Following the Euclidean algorithm, we now divide our quotient, which is $p(x)$ with our remainder. So we have:
\begin{center}
$\polylongdiv{x^3 - 9x^2 + 23x - 15}{5x^2 - 36x + 55}$\\
$ \implies p(x) = \frac{1}{5} \cdot (x - \frac{9}{5}) \cdot (5x^2 - 36x + 55) - \frac{24}{25} \cdot (x-5)$
\end{center}
Since we still have a remainder in $(x-5)$ we are still not done, so we again divide quotient with the remainder and get:
\begin{center}
$\polylongdiv{5x^2 - 36x + 55}{x-5}$\\
$\implies (5x^2 - 36x + 55) = (5x - 11) \cdot (x-5)$
\end{center}
Since now there is no remainder and only quotient, we know that according to Euclidean algorithm our $gcd(p,q)= (x-5)$. Now all that is left is to express it in the monic form. Since we have our two factorized equations:
\begin{center}
$q(x) = 1 \cdot p(x) - (5x^2 - 36x + 55)$\\
$p(x) = \frac{1}{5} \cdot (x - \frac{9}{5}) \cdot (5x^2 - 36x + 55) - \frac{24}{25} \cdot (x-5)$
\end{center}
We can express $(x-5)$ from there as:
\begin{center}
$(x-5) = -\frac{25}{24}  \cdot p(x) + \frac{5}{24} \cdot (x - \frac{9}{5}) \cdot (5x^2 - 36x + 55)$	
\end{center}
From the equation for $q(x)$ we have that $(5x^2 - 36x + 55) = p(x) - q(x)$, so we can further substitute and get:
\begin{center}
$(x-5) = -\frac{25}{24} \cdot p(x) + \frac{5}{24} \cdot (x - \frac{9}{5}) \cdot (p(x)-q(x))$\\
$\implies (x-5) = -\frac{25}{24} \cdot p(x) + \frac{5}{24} \cdot (x - \frac{9}{5}) \cdot p(x) - \frac{5}{24} \cdot (x - \frac{9}{5}) \cdot q(x)$ \\
$\implies (x-5) = \frac{5}{24} \cdot (-5 + x - \frac{9}{5}) \cdot p(x) - \frac{5}{24} \cdot (x - \frac{9}{5}) \cdot q(x)$\\
$\implies (x-5) = \frac{5}{24} \cdot (x - \frac{34}{5}) \cdot p(x) - \frac{5}{24} \cdot (x - \frac{9}{5}) \cdot q(x)$\\
$\implies (x-5) = \frac{1}{24} \cdot (5x - 34) \cdot p(x) - \frac{1}{24} \cdot (5x - 9) \cdot q(x)$
\end{center}
So the monic form is finally equal to:
\begin{center}
$ (x-5) = \frac{1}{24} \cdot (5x - 34) \cdot p(x) - \frac{1}{24} \cdot (5x - 9) \cdot q(x)$
\end{center}
Where polynomials $u$ and $v$ are $u=\frac{1}{24}(5x - 34)$ and $v=-\frac{1}{24}(5x - 9)$ over a ring of rational numbers $\mathbb{Q}[x]$.
\end{proof}
\end{document}