\documentclass[a4paper]{article}

\usepackage{amsmath,amsthm,amssymb, titling, tikz, mdframed}

\title{\vspace{-2cm}Lecture XII\vspace{-2cm}}
\date{}

\newmdtheoremenv{theorem}{Theorem}
\newmdtheoremenv{definition}{Definition}
\newmdtheoremenv{corollary}{Corollary}
\newmdtheoremenv{algorithm}{Algorithm}
\newmdtheoremenv{lemma}{Lemma}

\begin{document}
\maketitle
\section{Lecture Summary}
Speed-up of the parallel algorithm: $SU(n) =_{df} \frac{T_{seq}(n)}{T_{par}(n)}$, \\
Efficiency of the parallel algorithm: $EFF(n) =_{df} \frac{SU(n)}{Pr(n)}$. \\
Where $Pr(n)$ is number of processors, $T_{seq}(n)$ is sequential time of executing operation and $T_{par}(n)$ is parallel time.
\begin{definition}
\textbf{Boolean Circuit -} A Boolean circuit $C$ with $n$ inputs and $m$ outputs is a finite acyclic direct graph whose nodes are marked as follows: All nodes with input valence $0$ are marked by $x_{1}, \ldots, x_{n}$ or constants $0$, $1$ (input nodes). Nodes with input valence $1$ are marked by $\lnot$ (logical negation). The nodes with input valence $2$ are either marked by $\land$ or $\lor$ (logical AND and OR, respectively). There are no nodes having an input valence greater than $2$. Exactly $m$ nodes have output valence $0$ and they are marked by $y_{1}, \ldots, y_{m}$ (output nodes). \\
It is assumed that there is at least one path from every input node to an output node. \\
Such circuit $C$ computes a function $f: \{ 0, 1 \}^{n} \rightarrow \{ 0, 1 \}^{m}$ in the following way: Input $x_{i}, 1 \leq i \leq n$, is assigned a value $\nu (x_{i})$ from $\{ 0, 1 \}$ representing it $i$th bit of the argument of function $f$. Every other node $\nu$ is assigned a value $\nu (v) \in \{ 0, 1 \}$ obtained by applying the logical function node $v$, and is marked with to the value(s) of the nodes incoming to $v$. The value of the function is the tuple $(\nu(y_{1}), \ldots, \nu(y_{m}))$, in which output node $y_{j}$, where $1 \leq j \leq m$, contributes the $j$th bit of the output.
\end{definition}


\begin{definition}
For measuring the complexity of a Boolean circuit $C$ we use the depth of $C$ defined as the length of the longest path from an input node to an output node (parallel time) as well as the size of $C$ given by the total number of nodes (parallel amount of hardware).
\end{definition}


\begin{definition}
\textbf{Boolean Circuit Family -} A Boolean circuit family $(C_{n})_{n \in \mathbb{N}}$ is a sequence of circuits, where each $C_{n}$ computes the function $f_{C_{n}}: \{ 0, 1 \}^{n} \rightarrow \{ 0, 1 \}^{m(n)}$. The function $f$ computed by $(C_{n})_{n \in \mathbb{N}}$ is defined as
$$f_{C}: \{ 0, 1 \}^{*} \rightarrow \{ 0, 1 \}^{*}$$
where $f_{C}(x) =_{df} f_{C_{|x|}}(x)$.
\end{definition}


\begin{samepage}
\begin{definition}
Let $k \in \mathbb{N}$; then the class $\mathcal{N}\mathcal{C}^{k}$ is the set of all languages $L$ such that $L$ is accepted by a log-space uniform family of Boolean circuits simultaneously having depth $O((\log{n})^{k})$ and polynomial size. \\
The class $\mathcal{N}\mathcal{C}$ is the collection of all $\mathcal{N}\mathcal{C}^{k}$, i.e.,
$$\mathcal{N}\mathcal{C} = \bigcup\limits_{k \in \mathbb{N}} \mathcal{N}\mathcal{C}^{k}.$$
\end{definition}
\end{samepage}


\begin{theorem}
Let $p_{i} = a_{i} \otimes b_{i}$ and $g_{i} = a_{i} \land b_{i}$ for all $i = 0, \ldots, n - 1$. Furthermore let
$$(r_{i}, t_{i}) = (p_{0}, g_{0})\ \tikz\draw[black,fill=black] (0,0) circle (.5ex);\ (p_{1}, g_{1})\ \tikz\draw[black,fill=black] (0,0) circle (.5ex);\ \cdots\ \tikz\draw[black,fill=black] (0,0) circle (.5ex);\ (p_{i}, g_{i});$$
then we have $c_{i + 1} = t_{i}$.
\end{theorem}


\begin{theorem}
For $n \geq 2$ all the carries in an $n$-bit binary addition can be computed in time proportional to $\log{n}$ and in area proportional to $n \log{n}$, and so can the binary addition; i.e., binary addition is in $\mathcal{N}\mathcal{C}^{1}$.
\end{theorem}



\section{Exercises}
\noindent \textbf{Exercise 12 a)}  Non-uniform circuit families are unexpectedly powerful, since they can also “compute” non-computable functions like the halting problem.
\begin{proof}
The halting problem represents the problem of determining whether the program will finish running or run forever, depending from the description of the program. If we have a language with inputs of arbitrary size we can treat it by a family of circuits, one for each size of input. If these circuits are non-uniform, such that here is no Turing Machine which on input $n$ outputs the circuit for size $n$ inputs, so it allows circuit to solve uncomputable problems, giving answer for a certain encoding of the halting problem.
\end{proof}



\noindent\rule{12cm}{0.4pt}\\
\noindent \textbf{Exercise 12 b)} \tikz\draw[black,fill=black] (0,0) circle (.5ex); is associative. i.e.:
\begin{center}
$[(p_{0},g_{0}) \tikz\draw[black,fill=black] (0,0) circle (.5ex); (p_{1},g_{1})] \tikz\draw[black,fill=black] (0,0) circle (.5ex); (p_{2},g_{2}) = (p_{0},g_{0}) \tikz\draw[black,fill=black] (0,0) circle (.5ex); [(p_{1},g_{1}) \tikz\draw[black,fill=black] (0,0) circle (.5ex); (p_{2},g_{2})]$
\end{center}
\begin{proof}
Since the definition of \tikz\draw[black,fill=black] (0,0) circle (.5ex); relies on logical conjunction and logical disjunction ($\left(P_{1},G_{1}\right) \tikz\draw[black,fill=black] (0,0) circle (.5ex); \left(P_{2},G_{2}\right) =_{df} \left(P_{1} \land P_{2}, G_{2} \lor \left(P_{2} \land G_{1} \right) \right)$, we will prove this through using properties of logical operators. It is important to note before we start that both conjunction and disjunction are associative (for example, $(a \land b) \land c = a \land (b \land c)$. We will also use distributive law:
\begin{center}
$(a \lor (b \land c)) = ((a \lor b) \land (a \lor c))$\\
$(a \land (b \lor c)) = ((a \land b) \lor (a \land c))$\\
\end{center}
\begin{samepage}
And law of absorbtion:
\begin{center}
$(a \lor (a \land b)) =a$\\
$(a \land (a \lor b)) =a$\\
\end{center}
\end{samepage}
To begin we start with the left hand side of the equation:
\begin{center}
$[(p_{0},g_{0}) \tikz\draw[black,fill=black] (0,0) circle (.5ex); (p_{1},g_{1})] \tikz\draw[black,fill=black] (0,0) circle (.5ex); (p_{2},g_{2}) $\\
$\left(p_{0} \land p_{1}, g_{1} \lor \left(p_{1} \land g_{0} \right) \right) \tikz\draw[black,fill=black] (0,0) circle (.5ex); (p_{2},g_{2})$ (definition of \tikz\draw[black,fill=black] (0,0) circle (.5ex);)\\
$\left(p_{0} \land p_{1} \land p_{2}\right), \left( g_{2} \lor \left(p_{2} \land \left(g_{1} \lor \left(p_{1} \land g_{0} \right) \right) \right) \right)$ (definition of \tikz\draw[black,fill=black] (0,0) circle (.5ex);)
\end{center}
for the time being we will mark $( 1 ) = g_{2} \lor \left(p_{2} \land \left(g_{1} \lor \left(p_{1} \land g_{0} \right) \right) \right)$ and take a look at the right hand sign of the starting equation:
\begin{center}
$(p_{0},g_{0}) \tikz\draw[black,fill=black] (0,0) circle (.5ex); [(p_{1},g_{1}) \tikz\draw[black,fill=black] (0,0) circle (.5ex); (p_{2},g_{2})] $\\
$(p_{0},g_{0}) \tikz\draw[black,fill=black] (0,0) circle (.5ex); \left(p_{1} \land p_{2}, g_{2} \lor \left(p_{2} \land g_{1} \right) \right)$ (definition of \tikz\draw[black,fill=black] (0,0) circle (.5ex);)\\
$\left(p_{0} \land p_{1} \land p_{2}\right), \left( \left( g_{2} \lor \left(p_{2} \land g_{1} \right) \right) \lor \left(\left(p_{1} \land p_{2}\right) \land g_{0} \right) \right)$ (definition of \tikz\draw[black,fill=black] (0,0) circle (.5ex);)
\end{center}
We can see that the expression to the left of , is same in both cases, so we will take the expression to the right of it and apply distributive and absorption laws, and after simplifying the expression we will get that it equals to $g_{2} \lor \left(p_{2} \land \left(\left(g_{1} \lor p_{1} \right) \land \left( g_{1} \lor g_{0} \right) \right) \right)$, which after applying the distributive law, we get exactly ( 1 ), thus we can say that associative property holds for \tikz\draw[black,fill=black] (0,0) circle (.5ex);.
\end{proof}

\end{document}